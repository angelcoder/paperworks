\documentclass[11pt]{article}
\usepackage{blindtext}
\usepackage{titlesec}
\usepackage{epigraph}
\usepackage[utf8]{inputenc}
\usepackage[english, ukrainian]{babel}
\usepackage{amsmath, amssymb}
\usepackage[top = 1 cm, left = 2 cm, right = 1 cm, bottom = 2 cm]{geometry} 
\ifx\pdfoutput\undefined
\else
\usepackage[dvips]{graphicx}
\fi
\usepackage{graphicx}
\usepackage{amsthm}

\newcommand{\NN}{\mathbb{N}} 
\newcommand{\ZZ}{\mathbb{Z}}
\newcommand{\QQ}{\mathbb{Q}}
\newcommand{\RR}{\mathbb{R}}
\newcommand{\CC}{\mathbb{C}}

\newcommand{\Max}{\displaystyle\max\limits}
\newcommand{\Min}{\displaystyle\min\limits}
\newcommand{\Sum}{\displaystyle\sum\limits}
\newcommand{\Int}{\displaystyle\int\limits}
\newcommand{\Lim}{\displaystyle\lim\limits}
\newcommand{\Prod}{\displaystyle\prod\limits}


\allowdisplaybreaks

\setlength{\parindent}{0pt}


\date{}
\usepackage[dvipsnames]{xcolor}
\usepackage{colortbl}
\usepackage{amsmath}
\usepackage{esvect}


\addto\captionsenglish{
    \renewcommand*\contentsname{Summary}
}

\begin{document}
\begin{titlepage}
\newpage

\begin{center}
 \large{\textbf{ОСНОВНІ ВИЗНАЧЕННЯ ТА ТЕОРЕМИ З КУРСУ МАТЕМАТИЧНОГО АНАЛІЗУ ДЛЯ СТУДЕНТІВ ФАКУЛЬТЕТУ КОМП'ЮТЕРНИХ НАУК ТА КІБЕРНЕТИКИ КИЇВСЬКОГО НАЦІОНАЛЬНОГО УНІВЕРСИТЕТУ ІМЕНІ ТАРАСА ШЕВЧЕНКА}} \\
\end{center}

\vspace{8em}

\begin{center}
\large{Посібник з підготовки до модуля за одну ніч `\tt{:)}}
\end{center}

\vspace{5em}
\begin{center}
\large{НА ОСНОВІ ЛЕКЦІЙНОГО МАТЕРІАЛУ НОМІРОВСЬКОГО Д. А.} \\ 
\end{center}
\begin{center}

\vspace{8em}

\includegraphics[origin = 10]{images}
\end{center}

\vspace{10em}

\begin{flushright}
Виконали студенти групи К-23 

{\textbf{\textit{Логвіна А.В.}}} та {\textbf{\textit{Калюжний Д.В.}}}
\end{flushright}

\vspace{\fill}

\begin{center}
Київ \\2018
\end{center}

\end{titlepage}
\tableofcontents

\newpage
\section{\Large{Вступ}}

\subsection{\large{Відношення та функції}}

\begin{definition} 
    Нехай $X, Y \: -$ множини. \textcolor{NavyBlue}{\textbf{\textit{Декартовим добутком}}} $X \times Y$ називають множину усiх упорядкованих пар $(x; \: y)$, де $x \in X$ та $y \in Y.$
\end{definition}

\begin{definition} 
    Нехай $X \times Y \: -$ декартів добуток, $\Gamma\subset X \times Y$. Тоді $\Gamma$ називають \textcolor{NavyBlue}{\textbf{\textit{відношенням між \newline множинами}}} X i Y. 
\end{definition}

\begin{definition} 
        Нехай $\Gamma\subset X \times Y \: - $ відношення. Відношення $\Gamma$ називають \textcolor{NavyBlue}{\textbf{\textit{функціональним (функція)}}}, якщо $(x; \: y_1) \in \Gamma \wedge  (x; \: y_2) \in \Gamma \Rightarrow y_1 = y_2$.
\end{definition}

\begin{definition} 
        Нехай $\Gamma\subset X \times Y$ - відношення. Тоді відношення $\Gamma^{-1} = \left \{(y; \: x)|\:(x ; \: y) \in \Gamma \right \} \subset Y \times X$ \textcolor{NavyBlue}{\textbf{\textit{обернене}}} до відношення $\Gamma$.
\end{definition}

\begin{definition} 
        Нехай $\Gamma\subset X \times Y \: -$ функціональне відношення. Якщо $\Gamma^{-1}\subset Y \times X$ також є функціональним, то його $(\Gamma^{-1})$ називають \textcolor{NavyBlue}{\textbf{\textit{оберненою функцією}}} до функції $\Gamma$.
\end{definition}

\begin{theorem}[\textcolor{Maroon}{\textbf{\textit{Кантора}}}]
       Нехай $[a_1; \: b_1] \supset  [a_2; \: b_2] \supset...\supset[a_n; \: b_n] \supset ... ; \; a_i, \: b_i \in \mathbb{R}$. Тоді $\exists x \in \mathbb{R}: \ \forall i \in \mathbb{N} \; x \in [a_i; \: b_i]$. 
\end{theorem}

\begin{theorem}
    Дійсні числа відрізку  $[0; \: 1]$  не можна перелічити.
\end{theorem}

\subsection{\large{Топологія дійсних чисел}}

\begin{definition} 
        \textcolor{NavyBlue}{\textbf{\textit{Топологія}}}$ \: -$ сукупність всіх відкритих множин.
\end{definition}

\begin{definition} 
        Нехай $x_0 \in \mathbb{R},\: \varepsilon \in \mathbb{R}, \: \varepsilon > 0$. Множину $(x_0 - \varepsilon; \: x_0 + \varepsilon)$ називають \textcolor{NavyBlue}{\textbf{\textit{$\varepsilon$-околом точки}}} $x_0$.
\end{definition}

\begin{definition} 
        Нехай $x_0 \in \mathbb{R}$.\; Мн.  $M \subset \mathbb{R}$ називають \textcolor{NavyBlue}{\textbf{\textit{околом точки}}} $x_0$, якщо $\exists \varepsilon > 0$, що $(x_0 - \varepsilon; \: x_0 + \varepsilon) \subset M$.
\end{definition}



\begin{definition} 
       Нехай $M \subset \mathbb{R}$. Множина $M$ називається \textcolor{NavyBlue}{\textbf{\textit{відкритою}}}, якщо вона є околом кожної своєї точки.
\end{definition}

\begin{definition} 
       Нехай $M \subset \mathbb{R}$. Сукупність всіх внутрішніх точок множини $M$ називають \textcolor{NavyBlue}{\textbf{\textit{внутрішністю}}}\;$M$ і позначають $intM$.
\end{definition}

\begin{definition} 
       Нехай $x_0 \in \mathbb{R},\: M \subset \mathbb{R} $. Кажуть, що $x_0 \: -$  \textcolor{NavyBlue}{\textbf{\textit{точка дотику множини}}} $M$, якщо у будь-якому околі точки $x_0$ є елементи множини $M$.
\end{definition}

\begin{definition} 
       Множину $M \subset \mathbb{R}$ називають \textcolor{NavyBlue}{\textbf{\textit{замкненою}}}, якщо вона містить всі свої точки дотику.
\end{definition}

\begin{definition} 
       Нехай $M \subset \mathbb{R}$. Сукупність всіх точок дотику множини $M$ називають \textcolor{NavyBlue}{\textbf{\textit{замиканням}}}\;$M$ \newline і позначають $clM$.
\end{definition}
%%%%%%%%%%%%%%%%%%%%%%%%%%%%%%%%%%%%%%%%%%%%%%%%%%%%%%%%%%%%%%%%%%%%%%%%%%%%%%%%%%

\section{\Large{Послідовності}}
\subsection{\large{Границя числової послідовності}} 
\begin{definition} 
        Функцію $\mathbb{N} \xrightarrow[\text{}]{\text{f}}
        \mathbb{R} $ називають \textcolor{NavyBlue}{\textbf{\textit{числовою послідовністю}}}.
\end{definition}
\begin{definition}[топологічне]
        Число $a \in \mathbb{R}$ називається \textcolor{NavyBlue}{\textbf{\textit{границею послідовності}}} $x_n$, якщо\newline  будь-який окіл точки $a$ містить всі члени послідовності $x_n$, окрім, можливо, скінченної кількості.
\end{definition}

\begin{definition}[кванторне]
        Число $a \in \mathbb{R}$ називається \textcolor{NavyBlue}{\textbf{\textit{границею послідовності}}} $x_n$, якщо  \newline \;  $\forall \varepsilon > 0 \; \; \exists N: \ \forall n > N \; \; \left| x_n - a \right| < \varepsilon$.
\end{definition}

\begin{theorem}[\textcolor{Maroon}{\textbf{\textit{про єдиність границі}}}]
        Нехай $x_n \to a$, $x_n \to b$. Тоді $a = b$.  
\end{theorem}

\begin{definition} 
       Послідовність $x_n$ називають \textcolor{NavyBlue}{\textbf{\textit{обмеженою}}}, якщо існують сталі $c_1$ та $c_2$, що   $\forall n \in \mathbb{N}$  $c_1 < x_n < c_2$.
\end{definition}

\begin{theorem}[\textcolor{Maroon}{\textbf{\textit{необхідна умова збіжності}}}]
        Якщо $x_n \: -$ збіжна послідовність, то $x_n \: -$ обмежена.  
\end{theorem}

\begin{theorem}
        Нехай $x_n \to a, \; y_n \to b\:$  та  $\:x_n \leqslant  \: y_n$  $\forall n \in \mathbb{N}$. Тоді $a\leqslant \:b$.
\end{theorem}

\begin{theorem}
        Нехай $x_n \to a, \; y_n \to b\:$  та
        $a < \: b$. Тоді  $x_n <  \: y_n$  $\forall n \geqslant N_0$.  
\end{theorem}

\begin{theorem}[\textcolor{Maroon}{\textbf{\textit{про двох поліцаїв}}}]
        Нехай $x_n,\; y_n,\; z_n \: -$ послідовності та $x_n \leqslant  y_n \leqslant  z_n \;\: \forall n \in \mathbb{N}$. \newline Якщо $x_n \to a$ і $z_n \to a$, то $y_n$ є збіжною і $y_n \to a$.
\end{theorem}
\subsection{\large{Символи Ландау}}
\begin{definition} 
       Якщо $x_n \to 0$, то пишуть $x_n = o(1)$. Приклад: $\dfrac{1}{n} = o(1)$, але  $o(1) \neq \dfrac{1}{n}$.
       \end{definition}
 \begin{definition}       
       Якщо $x_n -$ обмежена посл., то пишуть $x_n = O(1)$. Приклад: $(-1)^{n} = O(1)$, але $O(1) \neq (-1)^{n}$.
       \end{definition}
       Властивості:
       \begin{enumerate}
           \item $o(1) + o(1) = o(1)$
           \item $O(1) + O(1) = O(1)$
           \item $o(1) + O(1) = O(1)$
           \item $o(1) \cdot o(1) = o(1)$
           \item $o(1) \cdot O(1) = o(1)$
           \item $O(1) \cdot O(1) = O(1)$
       \end{enumerate}


\begin{theorem}[\textcolor{Maroon}{\textbf{\textit{про структуру збіжних послідовностей}}}]
        Послідовність $x_n \to a$ $\iff$ її можна подати у вигляді $x_n = a + o(1)$.
\end{theorem}

\begin{theorem}[\textcolor{Maroon}{\textbf{\textit{{{про арифметичні дії зі збіжними послідовностями}}}}}]
        Нехай $x_n$, $\:y_n -$ збіжні. $x_n \to a$, $y_n \to b$. Тоді послідовності $x_n \pm y_n$, $x_n \cdot y_n$, $\dfrac{x_n}{y_n} \; (y_n \neq 0, \; b \neq 0) $ будуть також збіжні та
         \begin{enumerate}
        \item $x_n \pm y_n \to a \pm b$
        \item $x_n \cdot y_n \to a \cdot b$
        \item $\dfrac{x_n}{y_n} \to \dfrac{a}{b}$
         \end{enumerate}
\end{theorem}
\subsection{\large{Теорема Вейєрштраcса та число $e$}}

\begin{theorem}[\textcolor{Maroon}{\textbf{\textit{{{Вейєрштрасса}}}}}]
        Нехай $x_n \: -$ монотонна і обмежена послідовність, тоді вона є збіжною.
\end{theorem}

\begin{definition} 
       Нехай $M \subset \mathbb{R}$. Число $a$ називають \textcolor{NavyBlue}{{\textbf{\textit{супремумом множини}}}} $M$, якщо
        \begin{enumerate}
       \item  $\forall x \in M \ x \leqslant  a$ ($a \:- $ мажоранта множини $M$)
       \item  $\forall b < a \ \exists x \in M: x > b $ ($a \:- $ найменша з мажорант множини
       $M$)
        \end{enumerate}
\end{definition}

\begin{definition} 
       Нехай $M \subset \mathbb{R}$. Число $a$ називають \textcolor{NavyBlue}{\textbf{\textit{{інфінумом}}}} множини $M$, якщо
        \begin{enumerate}
       \item  $\forall x \in M \ x \geqslant  a$ ($a \:- $ міноранта множини $M$)
       \item  $\forall b > a \ \exists x \in M: x < b $ ($a \:- $ найбільша з мінорант множини $M$)
        \end{enumerate}
\end{definition}

\begin{theorem}[\textcolor{Maroon}{\textbf{\textit{{{повнота за Вейєрштрассом}}}}}]
         Нехай $M \: - $ обмежена зверху множина, тоді існує супремум \newline множини $M$.
\end{theorem}

\begin{definition} 
       Границею послідовності $x_n = \left(1 + \dfrac{1}{n}\right)^{n}$ називають числом \textcolor{NavyBlue}{\textbf{\textit{{e $\approx$ 2.718281828}}}}.
\end{definition}

\begin{definition} 
       Будемо казати, що послідовність $x_n \in \mathbb{R}$ \textcolor{NavyBlue}{\textbf{\textit{{збігається до $+\infty$ в $\mathbb{\overline{R}}$}}}}, якщо \newline $\forall c > 0$  $\exists N$ $\forall n > N$ $x_n > c$.
\end{definition}

\subsection{\large{Підпослідовності}}
\begin{definition} 
       Нехай $x_n - $ послідовність. Викреслимо з $x_n$ декілька членів, наприклад \newline \centerline{$x_n: \; x_1, \; \not{x_2}, \; \not{x_3},\; x_4,\ \not{x_5},\; x_6,\; \not{x_7},\; x_8,\; x_9,\:...$.} Послідовність $y_n$ (утворена із членів, що залишились) називається \textcolor{NavyBlue}{\textbf{\textit{{підпослідовністю}}}} послідовності $x_n$. \newline Тобто $y_k = x_{n_k}$, де $n_k \: - $ строго зростаюча послідовність натуральних чисел. 
\end{definition}

\begin{theorem}[]
         Якщо $x_n - $ збіжна послідовність, то будь-яка її підпослідовність $x_{n_k}$ також є збіжною \newline і $\Lim_{k \to \infty} x_{n_k} = \Lim_{n \to \infty} x_n$. АЛЕ! Обернене твердження місця не має.
\end{theorem}

\begin{theorem}[\textcolor{Maroon}{\textbf{\textit{{{Больцано-Вейєрштрасса}}}}}]
         Будь-яка послідовність містить монотонну підпослідовність.
         \item \textcolor{Maroon}{\textbf{\textit{{{Наслідок}}}}}: Будь-яка обмежена послідовність містить збіжну підпослідовність.
\end{theorem}

\begin{definition} 
       Нехай $x_n -$ послідовність і $x_{n_k} -$ її збіжна підпослідовність, $x_{n_k} \to a$. Тоді число $a$ називають \textcolor{NavyBlue}{\textbf{\textit{{частковою границею}}}} послідовності $x_n$.
\end{definition}

\begin{definition}
       Найбільшу часткову границю посл. $x_n$ називають \textcolor{NavyBlue}{\textbf{\textit{{верхньою границею}}}} і позначають {$\overline{\Lim_{n \to \infty}} x_n $.}
\end{definition}

\begin{theorem}[\textcolor{Maroon}{\textbf{\textit{{{гармонічний ряд}}}}}]
         $\Sum_{k=1}^n \dfrac{1}{k} = \ln{n} + c + o(1) \;$, де $c \approx 0.57 \: -  $ стала Ейлера. 
\end{theorem}

\begin{definition}
         Послідовність $x_n$ називають \textcolor{NavyBlue}{\textbf{\textit{{фундаментальною}}}}, якщо $x_n - x_m \xrightarrow[n, m \to +\infty]{}0$, тобто \newline $\forall \varepsilon > 0 \ \exists N \  \forall n > N \ \forall m > N \ |x_n - x_m| < \varepsilon$.
\end{definition}

\begin{theorem}[\textcolor{Maroon}{\textbf{\textit{{{критерій Коші}}}}}]
         Послідовність збіжна $\iff $ вона фундаментальна.
\end{theorem}
\subsection{\large{Теорема Штольца}}
\begin{theorem}[\textcolor{Maroon}{\textbf{\textit{{{Тепліца}}}}}]
         Нехай $x_n \to a \in \mathbb{\overline{R}}$. Якщо числа \newline
\begin{equation}
      \label{ep:1.1}
      \begin{aligned}
          p_{11} \\
          p_{21} &, \ p_{22} \\
          p_{31} &, \ p_{32} , \ p_{33} \\
          p_{41} &, \ p_{42}  , \ p_{43} , \ p_{44} \\
           & \ldots\ldots\ldots\ldots    
        \end{aligned}
    \end{equation}
     задовольняють умови
      \begin{enumerate}
       \item $p_{ij} \geqslant  0$
       \item $p_{k1} + p_{k2} + \ldots + p_{kk} = 1, \ k \in \mathbb{N}$
       \item $\Lim_{k \to \infty} p_{kj} = 0, \ j \in \mathbb{N}$
        \end{enumerate}
        то послідовність $y_n = p_{n1}x_1 + p_{n2}x_2 + \ldots + p_{nn}x_n \to a$.
         \item \textcolor{Maroon}{\textbf{\textit{{{\textbf{\textit{Наслідок}}}}}}}: Нехай $x_n, \; y_n -$ послідовності і $y_n > 0$,  \ $\Sum_{k=1}^n y_k \to +\infty$ в $\mathbb{\overline{R}}$. Якщо $\dfrac{x_n}{y_n} \to a$ в $\mathbb{\overline{R}}$, то \newline $z_n = \dfrac{x_1 \; + \; x_2 \; + \;  \ldots \; + \; x_n}{y_1 \; + \; y_2  \; + \; \ldots \; + \; y_n} \to a$.
\end{theorem}

\begin{theorem}[\textcolor{Maroon}{\textbf{\textit{{{Штольца}}}}}]
        Нехай $x_n$, $y_n \: -$ послідовності. Якщо
        \begin{enumerate}
            \item $y_n \: -$ строго зростаюча послідовність
            \item $y_n \to +\infty$ в $\mathbb{\overline{R}}$
            \item $\dfrac{x_n \;- \; x_{n-1}}{y_n \; - \; y_{n-1}} \to a$ в $\mathbb{\overline{R}}$
        \end{enumerate}
        Тоді $\dfrac{x_n}{y_n} \to a$.
\end{theorem}

%%%%%%%%%%%%%%%%%%%%%%%%%%%%%%%%%%%%%%%%%%%%%%%%%%%%%%%%%%%%%%%%%%%%%%%%%%%%%%%%%%

\section{\Large{Функції}}

\subsection{\large{Неперервні функції}}

\begin{definition}[за Гейне]
         Нехай $f \: -$ функція, $x_0 \in D_f$. Будемо казати, що $f$ є \textcolor{NavyBlue}{\textbf{\textit{{неперервною}}}} в точці $x_0$, якщо  $\forall x_n \in D_f:$ $x_n \to x_0 \Rightarrow  f(x_n) \to f(x_0).$
\end{definition}

\begin{theorem}[\textcolor{Maroon}{\textbf{\textit{{{про арифметичні дії з неперервними функціями}}}}}]
       Нехай $f$ і $g \: -$ неперервні в точці $x_0$. \newline Тоді функції $f \pm g$, $f \cdot g$, $\dfrac{f}{g}$ $(g(x_0) \neq 0)$ є неперервними в точці $x_0.$ 
\end{theorem}

\begin{theorem}[\textcolor{Maroon}{\textbf{\textit{{{про неперервність композиції неперервних функцій}}}}}]
       Нехай $f \:-$ неперервна в точці $x_0$, \newline а $g \: - $ в точці $y_0 = f(x_0).$ Тоді $g \circ f \:-$ неперервна функція в точці $x_0$.
\end{theorem}

\begin{definition}[за Коші]
      Функцію $f$ називають \textcolor{NavyBlue}{\textbf{\textit{{неперервною}}}} в точці $x_0 \in D_f$,  якщо \newline  $\forall \varepsilon > 0  \ \ \exists \delta > 0 \  \ \forall x \in D_f: \ |x - x_0| < \delta \Rightarrow |f(x) - f(x_0)| < \varepsilon.$          
\end{definition}
      
\begin{definition}[за Гейне]
      Нехай $f \: -$ функція, $x_0 \in D_f$. Будемо казати, що $f$ є \textcolor{NavyBlue}{\textbf{\textit{{неперервною справа}}}} в точці $x_0$, якщо  $\forall x_n \in D_f$, $x_n \geqslant  x_0$,  $x_n \to x_0 \Rightarrow  f(x_n) \to f(x_0).$      
\end{definition}
      
\begin{definition}[за Коші]
      Функцію $f$ називають \textcolor{NavyBlue}{\textbf{\textit{{неперервною зліва}}}} в точці $x_0 \in D_f$,  якщо \newline $\forall \varepsilon > 0  \ \ \exists \delta > 0 \  \ \forall x \in D_f: \ |x - x_0| < \delta$, $ x \leqslant  x_0 \Rightarrow |f(x) - f(x_0)| < \varepsilon.$          
\end{definition}

\begin{theorem}
       Функція $f$ є неперервною в точці $x_0 \iff$ вона є неперервною справа і зліва в точці $x_0.$
\end{theorem}

\subsection{\large{Границя функції в точці}}

\begin{definition}
      Точку $x_0 \in \mathbb{R}$ називають \textcolor{NavyBlue}{\textbf{\textit{{граничною точкою множини}}}} $M$, якщо існує послідовність різних чисел $x_n \in M$, що $x_n \to x_0.$
\end{definition}

\begin{definition}[за Гейне]
      Нехай $f \:- $ функція і $x_0 \:- $ гранична точка $D_f.$ Число $a \in \mathbb{R}$ називають \textcolor{NavyBlue}{\textbf{\textit{{границею функції}}}} $f$ в точці $x_0$, якщо  $\forall x_n \in D_f$, $x_n \neq x_0$, $x_n \to x_0 \Rightarrow f(x_n) \to a.$ Позначають $\Lim_{x \rightarrow x_0} f(x)=a$.
\end{definition}

\begin{theorem}[\textcolor{Maroon}{\textbf{\textit{про арифметичні дії з границями}}}]
       Нехай $x_0 \:-$ гранична точка множини $D_f \cap D_g$ і $\Lim_{x \to x_0} f(x) = A$, $\Lim_{x \to x_0} g(x) = B$. Тоді існують границі
       \begin{enumerate}
           \item $\Lim_{x \to x_0} (f(x) \pm g(x)) = A \pm B$
           \item $\Lim_{x \to x_0} (f(x) \cdot g(x)) = A \cdot B$
           \item $\Lim_{x \to x_0} \dfrac{f(x)}{g(x)} = \dfrac{A}{B}$ \ \ ($B \neq 0$)
       \end{enumerate}
\end{theorem}

\begin{theorem}[\textcolor{Maroon}{\textbf{\textit{{{\textbf{\textit{{{про двох поліцаїв}}}}}}}}}]
       Нехай $x_0 \: -$ гранична точка $D_f \cap D_g \cap D_h$ і  $\forall x \in D_f \cap D_g \cap D_h$ виконується $f(x) \leqslant  g(x) \leqslant  h(x).$ Якщо $\exists \Lim_{x \to x_0} f(x) = \Lim_{x \to x_0} h(x) = A$, то $\exists  \Lim_{x \to x_0} g(x) = A.$
\end{theorem}

\begin{theorem}[\textcolor{Maroon}{\textbf{\textit{{{\textbf{\textit{{{про границю композиції функцій №1}}}}}}}}}]
       Нехай $\Lim_{x \to x_0} g(x) = a$ і $f \: - $ неперервна функція в точці $a$. Якщо $x_0 \: - $ гранична точка $D_{f \circ g}$, то $\Lim_{x \to x_0}f(g(x)) = f(a).$
\end{theorem}

\begin{theorem}[\textcolor{Maroon}{\textbf{\textit{{{\textbf{\textit{{{про границю композиції функцій №2}}}}}}}}}]
      Нехай $x_0 \: - $ гранична точка $D_{f \circ g}$. Якщо
      \begin{enumerate}
            \item $\Lim_{x \to x_0} g(x) = a$
            \item $\Lim_{t \to a} f(t) = b$
            \item $\exists \ O(x_0)$, що  $\forall \: x \in O(x_0) \setminus \left \{ x_0 \right \}: \ x \in D_g \Rightarrow g(x) \neq a$
      \end{enumerate}
      Тоді $\exists \Lim_{x \to x_0} f(g(x)) = b$.

\end{theorem}
\subsection{\large{Невласні та односторонні границі}}
\begin{definition}[за Гейне] 
        Нехай $x_0 \: - $ гранична точка $D_f$. Кажуть, що функція $f$ має \textcolor{NavyBlue}{\textbf{\textit{{границю рівну $+\infty$ }}}}в точці $x_0$ в $\mathbb{\overline{R}}$, якщо  $\forall x_n \in D_f$: $x_n \to x_0,\; x_n \neq x_0 \Rightarrow f(x_n) \to +\infty$ 
        в $\mathbb{\overline{R}}$. Позначають $\Lim_{x \to x_0} f(x) = +\infty.$
\end{definition}

\begin{definition}[за Гейне] 
        Нехай існує $x_n \in D_f: x_n \rightarrow -\infty$ в $\mathbb{\overline{R}}$. Кажуть, що число $a$ є \textcolor{NavyBlue}{\textbf{\textit{{границею функції $f$ \newline  на $-\infty$}}}}, якщо  $\forall x_n \in D_f: \; x_n \to -\infty$ в $\mathbb{\overline{R}} \Rightarrow f(x_n) \to a$. Позначають $\Lim_{x \to -\infty} f(x) = a.$
\end{definition}

\begin{definition}[за Гейне] 
        Нехай $x_0 \: - $ гранична точка $D_f \cap (x_0; \: +\infty)$. Число $a \in \mathbb{R}$ називають \textcolor{NavyBlue}{\textbf{\textit{{границею функції}}}} $f$ \textcolor{NavyBlue}{\textbf{\textit{{справа}}}} в точці $x_0$, якщо  $\forall x_n \in D_f$: $x_n > x_0$, $x_n \to x_0 \Rightarrow f(x_n) \to a.$ Позначають $\Lim_{x \to x_0^{+}} f(x) = a$.
\end{definition}

\begin{theorem}
        Нехай $x_0 \: - $ гранична точка $D_f$. Тоді функція $f$ є неперервною в точці $x_0 \iff \Lim_{x \to x_0}f(x) = f(x_0)$.
\end{theorem}

\begin{definition}[за Коші]
         Нехай $x_0 \: - $ гранична точка $D_f$. Число $a \in \mathbb{R}$ називають \textcolor{NavyBlue}{\textbf{\textit{{границею функції}}}} $f$ в точці $x_0$, якщо  $\forall \varepsilon > 0 \ \ \exists \delta > 0: \ \forall x \in D_f: \; 0 < |x - x_0| < \delta \Rightarrow |f(x) - a| < \varepsilon$.    
\end{definition}

\begin{theorem}[\textcolor{Maroon}{\textbf{\textit{критерій Коші}}}]
        Нехай $x_0 \: - $ гранична точка $D_f$. \newline Тоді $f$ має границю в точці $x_0 \iff$  $\forall \varepsilon > 0 \ \exists \delta > 0 \ \: \forall x_1, x_2 \in D_f:$ 
        \begin{cases} 0 < |x_1 - x_0| < \delta  \\ 0 < |x_2 - x_0| < \delta  \end{cases} $\Rightarrow \ |f(x_1) - f(x_2)| < \varepsilon $.
\end{theorem}


\subsection{\large{Точки розриву функції}}

\begin{definition}
        Нехай $x_0 \: - $ гранична точка $D_f$ і рівність $\Lim_{x \to x_0}f(x) = f(x_0)$ не виконується ($f$ не є неперервною в точці $x_0$). Тоді кажуть, що $x_0 \: - $  \textcolor{NavyBlue}{\textbf{\textit{точка розриву}}} функцї $f$.   
\end{definition}

\begin{definition}
        Нехай $x_0 \: -$ точка розриву функції $f$ і $\exists \Lim_{x \to x_0} f(x)$. Тоді $x_0$ називають \textcolor{NavyBlue}{\textbf{\textit{{точкою усувного розриву}}}}.
\end{definition}

\begin{definition}
         Нехай $x_0 \: -$ точка розриву функції $f$ та існують $\Lim_{x \to x_0^{+}}f(x) \neq \Lim_{x \to x_0^{-}}f(x)$. Тоді $x_0$ називають \textcolor{NavyBlue}{\textbf{\textit{точкою розриву першого роду.}}}
\end{definition}

\begin{definition}
         Всі інші точки розриву називать \textcolor{NavyBlue}{\textbf{\textit{{точками розриву другого роду}}}}.
\end{definition}

\begin{theorem}
       Функція $f$ має границю в точці $x_0$, що дорівнює числу $a \iff f = a + o(1).$      
\end{theorem}

\subsection{\large{Властивості неперервних функцій}} 

\begin{theorem}[\textcolor{Maroon}{\textbf{\textit{{{про корінь неперервної функції}}}}}]
       Нехай $f \in C([a; \: b])$ та $f(a) f(b) \leqslant  0$. Тоді існує таке число $\xi \in [a; \: b]$, що $f(\xi) = 0$.
\end{theorem}

\begin{theorem}[\textcolor{Maroon}{\textbf{\textit{{{про проміжне значення неперервної функції}}}}}]
       Нехай $f\in C([a ; \: b])$, тоді $\forall l$ між числами \newline $f(a)$ і $f(b)$ існує число $\xi \in [a; \: b]$, що $f(\xi) = l$.
\end{theorem}

\begin{definition}
       Множину $K \subset \mathbb{R}$ називають \textcolor{NavyBlue}{\textbf{\textit{{компактом}}}}, якщо  $\forall x_n \in K  \ \: \exists x_{n_k}$, яка збігається до $x_0 \in K$.
\end{definition}

\begin{theorem}[\textcolor{Maroon}{\textbf{\textit{{{критерій компактності}}}}}]
       Множина $K \subset \mathbb{R}$ є компактом $\iff$ вона замкнена і обмежена.
\end{theorem}
        
\begin{theorem}[\textcolor{Maroon}{\textbf{\textit{про неперервний образ компакту}}}]
       Нехай $K \: -$ компакт, а функція $f \in C(K) \Rightarrow$ множина $f(K)$ також компакт.
\end{theorem}

\begin{theorem}[\textcolor{Maroon}{\textbf{\textit{Вейєрштрасса}}}]
       Будь-яка неперервна на компакті функція досягає свого максимуму та мінімуму. Як \textcolor{Maroon}{\textbf{\textit{{{наслідок}}}}} $-$ неперервна функція обмежена на компакті.
\end{theorem}

\subsection{\large{Рівномірно неперервні функції}} 

\begin{definition}[за Гейне]
         Нехай $M \subset D_f$. Кажуть, що функція $f$ \textcolor{NavyBlue}{\textbf{\textit{{рівномірно неперервна}}}} на множині $M$, якщо  $\forall  x_n, y_n \in M: x_n - y_n \to 0 \Rightarrow f(x_n) - f(y_n) \to 0$. 
\end{definition}

\begin{definition}[за Коші]
       Нехай $M \subset D_f$. Кажуть, що функція $f$ \textcolor{NavyBlue}{\textbf{\textit{{рівномірно неперервна}}}} на множині $M$, якщо  $\forall \varepsilon > 0 \ \exists \delta > 0 \ \forall x_1, x_2 \in M: |x_1 - x_2| < \delta \Rightarrow |f(x_1) - f(x_2)| < \varepsilon$.   
\end{definition}

\begin{definition}[за Коші]
        Нехай $M \subset D_f$. Кажуть, що функція $f$  \textcolor{NavyBlue}{\textbf{\textit{{неперервна}}}} \textcolor{NavyBlue}{\textbf{\textit{{на множині}}}} $M$, \newline якщо  $\forall x_1 \in M \ \forall \varepsilon > 0 \ \exists \delta > 0 \ \forall x_2 \in M: |x_1 - x_2| < \delta \Rightarrow |f(x_1) - f(x_2)| < \varepsilon$.   
\end{definition}

\begin{theorem}
        Якщо функція $f$ є рівномірно неперервна на множині $M$, то вона є неперервна на $M$.       
\end{theorem}

\begin{theorem}[\textcolor{Maroon}{\textbf{\textit{{{Гейне-Кантора}}}}}]
        Якщо $f$ неперервна на компакті $K$, то вона і рівномірно неперервна на $K$.
\end{theorem}

%%%%%%%%%%%%%%%%%%%%%%%%%%%%%%%%%%%%%%%%%%%%%%%%%%%%%%%%%%%%%%%%%%%%%%%%%%%%%%%

\section{\Large{Похідна}}

\subsection{\large{Диференційованість функцій}} 

\begin{definition}

         Нехай $x_0 \: - $ гранична точка $D_f$, $x_0 \in D_f$. Якщо існує $\Lim_{x \to x_0} \dfrac{f(x) - f(x_0)}{x - x_0}$, то кажуть, що функція $f$ \textcolor{NavyBlue}{\textbf{\textit{має  похідну (диференційована)}}} в точці $x_0$ і ${f}'(x_0) = \Lim_{x \to x_0} \dfrac{f(x) - f(x_0)}{x - x_0}$.
\end{definition}

\begin{definition}[за Ферма]
         Нехай $x_0 \: - $ гранична точка $D_f$, $x_0 \in D_f$. Кажуть, що $f$ \textcolor{NavyBlue}{\textbf{\textit{диференційована (має похідну)}}} в точці $x_0$, якщо існує така неперервна функція в точці $x_0 \ \varphi_f$, що $f(x) - f(x_0) = \varphi_f(x) (x - x_0)$  $\forall x \in D_f$. Тоді $f'(x_0) = \varphi_f(x_0)$.
\end{definition}

\begin{theorem}[\textcolor{Maroon}{\textbf{\textit{{{про арифметичні дії з диференційованими функціями}}}}}]
        Нехай $f$ і $g \: -$ диференційовані в точці $x_0$ функції та $D_f = D_g$. Тоді функції $f \pm g$, $f \cdot g$, $\dfrac{f}{g}$  $(g(x_0) \neq 0)$ також диференційовані в точці $x_0$ і
        \begin{enumerate}
            \item $(f \pm g)'(x_0) = f'(x_0) \pm g'(x_0)$
            \item $(f \cdot g)'(x_0) = f'(x_0) \cdot g(x_0) + f(x_0) \cdot g'(x_0)$
            \item $\left(\dfrac{f}{g}\right)'(x_0) = \dfrac{f'(x_0) \cdot g(x_0) - f(x_0) \cdot g'(x_0)}{g^2(x_0)}$
        \end{enumerate}
\end{theorem}

\begin{theorem}[\textcolor{Maroon}{\textbf{\textit{{{необхідна умова диференційованості}}}}}]
        Якщо функція диференційована в точці $x_0$, то вона неперервна в точці $x_0$.
\end{theorem}

\begin{theorem}[\textcolor{Maroon}{\textbf{\textit{{{про похідну складеної функції}}}}}]
        Нехай $f$ і $g \: - $ функції та $x_0 \: -$ гранична точка $D_{f \circ g}$, $x_0 \in D_{f \circ g}$. Якщо функція $g$ має похідну в точці $x_0$, а функція $f \: -$ в точці $t_0 = g(x_0)$, то $f \circ g$ має похідну в точці $x_0$ і $(f \circ g)'(x_0) = f'(t_0) g'(x_0).$
\end{theorem}

\begin{definition}
        Нехай $x_0 \: - $ гранична точка $D_f \cap (x_0; \: +\infty)$, $x_0 \in D_f$. $f_r'(x_0) = \Lim_{x \to x_0^{+}} \dfrac{f(x) - f(x_0)}{x - x_0}$ називають \textcolor{NavyBlue}{\textbf{\textit{{правою похідною}}}} функції $f$ в точці $x_0$.
\end{definition}

\begin{definition}
        Нехай $x_0 \: - $ гранична точка $D_f \cap (-\infty; \: x_0)$, $x_0 \in D_f$. $f_l'(x_0) = \Lim_{x \to x_0^{-}} \dfrac{f(x) - f(x_0)}{x - x_0}$ називають \textcolor{NavyBlue}{\textbf{\textit{{лівою похідною}}}} функції $f$ в точці $x_0$.
\end{definition}

\begin{definition}
        Нехай $f \: -$ диференційована в точці $x_0$ функція. Лінійну функцію $d_{x_0}f$, яку визначає рівність $d_{x_0}f(h) = f'(x_0) h$, називають \textcolor{NavyBlue}{\textbf{\textit{{диференціалом}}}} функції $f$ в точці $x_0$.
\end{definition}

\subsection{\large{Основні теореми диференціального числення}} 

\begin{definition}
        Точку $x_0 \in M$ називають \textcolor{NavyBlue}{\textbf{\textit{{точкою глобального мінімуму}}}} множини $M$, якщо \newline  $\forall x \in M \: f(x_0) \leqslant  f(x).$
\end{definition}

\begin{definition}
        Точку $x_0 \in M$ називають \textcolor{NavyBlue}{\textbf{\textit{{точкою глобального максимуму}}}} множини $M$, якщо \newline  $\forall x \in M \: f(x_0) \geqslant   f(x).$
\end{definition}

\begin{definition}
        Точку $x_0 \in M$ називають \textcolor{NavyBlue}{\textbf{\textit{{точкою локального мінімуму}}}} множини $M$, якщо \newline $\exists \: O(x_0)$, що  $\forall x \in O(x_0), \; x \in D_f \; f(x_0) \leqslant  f(x).$
\end{definition}

\begin{definition}
        Точку $x_0 \in M$ називають \textcolor{NavyBlue}{\textbf{\textit{точкою локального максимуму}}} множини $M$, якщо \newline $\exists \: O(x_0)$, що  $\forall x \in O(x_0), \; x \in D_f \; f(x_0) \geqslant  f(x).$
\end{definition}

\begin{theorem}[\textcolor{Maroon}{\textbf{\textit{{{Ферма}}}}}]
        Нехай $f$ має екстремум в точці $x_0$, яка є внутрішньою точкою $D_f$. \newline Якщо $f \: -$ диференційована в точці $x_0$, то $f'(x_0) = 0$.
\end{theorem}

\begin{theorem}[\textcolor{Maroon}{\textbf{\textit{{{Ролля}}}}}]
        Нехай $f \in C([a; \: b])$ та диференційована на $(a; \: b)$. Якщо $f(a) = f(b)$, то $\exists \xi \in (a; \: b)$, що $f'(\xi) = 0$.
\end{theorem}

\begin{theorem}[\textcolor{Maroon}{\textbf{\textit{{{Дарбу}}}}}]
        Нехай $f$ диференційована на $[a; \: b]$ і $f_r'(a)  f_l'(b) < 0$. Тоді $\exists \xi \in (a; \: b)$, що $f'(\xi) = 0$.
\end{theorem}

\begin{theorem}[\textcolor{Maroon}{\textbf{\textit{{{про проміжне значення похідної}}}}}]
         Нехай $f$ диференційована на $[a; \: b]$. Тоді $\forall l$ між числами $f'(a)$ і $f'(b) \ \exists \xi \in (a; \: b)$, що $f'(\xi) = l$.
\end{theorem}

\begin{theorem}[\textcolor{Maroon}{\textbf{\textit{{{Лагранжа}}}}}]
          Нехай $f \in C([a; \: b])$ та диференційована на $(a; \: b)$. Тоді $\exists \xi \in (a; \: b)$, що $f'(\xi) = \dfrac{f(b) - f(a)}{b - a}$.
\end{theorem}

\begin{theorem}[\textcolor{Maroon}{\textbf{\textit{{{про неперервність похідної}}}}}]
        Нехай $f$ неперервна в околі точки $x_0$ і  $\forall x \in O(x_0)\setminus \left \{ x_0  \right \} \ \exists \ f'(x)$. Якщо $\exists \Lim_{x \to x_0} f'(x) = a$, то $f$ має похідну і в точці $x_0$, тобто $f'(x_0) = a$.
\end{theorem}

\begin{theorem}[\textcolor{Maroon}{\textbf{\textit{{{про похідну оберненої функції}}}}}]
        Нехай $f$ має похідну в околі точки $x_0$, яка не обертається в нуль. Тоді в околі точки $P(x_0; \:  f(x_0))$ функція $f$ має обернену $f^{-1}$ і вона має похідну $(f^{-1})'(y_0) = \dfrac{1}{f'(x_0)}, \; y_0 = f(x_0)$.
\end{theorem}

\begin{theorem}[\textcolor{Maroon}{\textbf{\textit{{{Лейбніца}}}}}]
        Нехай $f$ і $g$ мають $n$-ту похідну на $(a; \: b)$. Тоді $f\cdot g$ має $n$-ту похідну на $(a; \: b)$ і  \centerline{$(f \cdot g)^{(n)} = \Sum_{k=0}^n C_n^{k}  f^{(k)}  g^{(n - k)}$}
\end{theorem}

\begin{definition}
        \textcolor{NavyBlue}{\textbf{\textit{ Другим диференціалом функції}}} $f$ в т. $x_0$ називають функцію, що визначається рівністью \centerline{$d^{2}_{x_0}f(h) = f''(x_0)  h^{2}$}
\end{definition}

\newpage
\section{\Large{Поняття інтеграла}}
\subsection{\large{Інтеграл Ньютона-Лейбніца}}


\begin{definition}
    Нехай $f(x):[a;\:b]\to\mathbb{R}$, тоді функцію $F$ називають                     \textcolor{NavyBlue}{\textbf{\textit{інтегралом Ньютона-Лейбніца}}} функції $f(x)$ на $[a;\:b]$, якщо 
    \begin{enumerate}
        \item $F(a) = 0$
        \item $F'(x) = f(x) \;\; \forall x \in [a; \: b]$
    \end{enumerate}
    Позначають: $F(x) = \Int_a^x f(t) \mathrm{d}t$ \\
\end{definition}


\subsection{\large{Інтеграл Рімана}}
\begin{definition}
    Набір $P = \{ x_0,\: x_1, \ldots, \: x_n \},\:$ де $a = x_0, \; b = x_n, \; x_i < x_{i + 1}$ називається \textcolor{NavyBlue}{\textbf{\textit{розбиттям відрізка}}} $[a;\:b]$.\\
\end{definition}


\begin{definition}
    Величину $d_P = \Max_{i = \overline{0, n - 1}} (x_{i + 1} - x_i)$ називають \textcolor{NavyBlue}{\textbf{\textit{діаметром розбиття}}} $[a;\:b]$. \\
\end{definition}


\begin{definition}
    Нехай є точка $\xi_i $ з проміжку $ [x_i;\:x_{i + 1})$, \textcolor{NavyBlue}{\textbf{\textit{інтегральною сумою Рімана}}} називають \\
    \[ S(P,\: \xi) = \Sum_{i=0}^{n-1} f(\xi_i) (x_{i+1} - x_i) = \Sum_{i=0}^{n-1} f(\xi_i) \Delta x_i \]
\end{definition}


\begin{definition}
    Число $I \in \RR$ називають \textcolor{NavyBlue}{\textbf{\textit{інтегралом Рімана функції}}} $f(x)$ на $[a;\:b]$, якщо
    \[ \forall \varepsilon>0\ \exists\delta>0 \ \forall P \ \forall\xi_i: \ d_P<\delta \Rightarrow \left| S(P,\:\xi) \right| <\varepsilon \]
\end{definition}


\subsection{\large{Інтеграл Дарбу}}


\begin{definition}
    Нехай $f(x)$ визначена та обмежена функція на $[a;\:b].$ \newline Суму $\underline{S}_P(f) = \Sum_{i=0}^{n-1} m_i\: \Delta x_i$\text{, де } $m_i = \underset{{x\in [x_i;\: x_{i+1}]}}{\inf} f(x)$ називають \textcolor{NavyBlue}{\textbf{\textit{нижньою сумою Дарбу}}}. \\
\end{definition}


\begin{definition}
    Нехай $f(x)$ визначена та обмежена функція на $[a;\:b].$ \newline
    Суму $\overline{S}_P(f) = \Sum_{i=0}^{n-1} M_i\:\Delta x_i$ , де $ M_i = \underset{{x\in [x_i;\: x_{i+1}]}}{\sup} f(x)$ називають \textcolor{NavyBlue}{\textbf{\textit{верхньою сумою Дарбу}}}. \\
\end{definition}


\begin{definition}
    Величину $\underline{S}_P(f)$ називають \textcolor{NavyBlue}{\textbf{\textit{нижнім інтегралом Дарбу}}} і позначають $\underline{\Int_a^b} f(x) \,\mathrm{d} x$. \\
\end{definition}


\begin{definition}
    Величину $\overline{S}_P(f)$ називають \textcolor{NavyBlue}{\textbf{\textit{верхнім інтегралом Дарбу}}} і позначають $\overline{\Int_a^b} f(x) \,\mathrm{d} x$. \\
\end{definition}


\subsection{\large{властивості сум Дарбу}}

\begin{definition}
    Нехай $P_1$ і $P_2 \: -$ деякі розбиття відрізка $[a;\:b]$. Кажуть, що $P_2\: -$  \textcolor{NavyBlue}{\textbf{\textit{продовження розбиття}}} $P_1$, якщо $P_1 \subset P_2$.
\end{definition}


\begin{theorem}[\textcolor{Maroon}{\textbf{\textit{властивості сум Дарбу}}}]
\end{theorem}
    \begin{enumerate}
        \item $\underline{S}_P(f) \leqslant \overline{S}_P(f)$
        \item Нехай $P_2\: -$  продовження розбиття $P_1$, тоді
        \begin{enumerate}
            \item $\underline{S}_{P_1}(f) \leqslant \underline{S}_{P_2}(f)$
            \item $\overline{S}_{P_1}(f) \geqslant \overline{S}_{P_2}(f)$
    \end{enumerate}
    \item $\forall P_1,\: P_2$: $\ \underline{S}_{P_1}(f) \leqslant \overline{S}_{P_2}(f)$
    \end{enumerate}


\begin{theorem}[\textcolor{Maroon}{\textbf{\textit{критерій інтегровності за Дарбу}}}]
    Нехай $f(x) \: -$ обмежена на $[a;\:b]$ функція. \newline Тоді $f(x)$ інтегровна за Дабру $\iff$  $\forall \varepsilon > 0 \ \exists P: \ \overline{S}_P(f) - \underline{S}_P(f) \leqslant \varepsilon$
\end{theorem}
\textcolor{Maroon}{\textbf{\textit{{Наслідок.}}}} Обмежена функція $f$ інтегровна за Дарбу $\iff$ інтегровна за Ріманом.


\subsection{\large{Властивості інтеграла Рімна}}


\begin{definition}
    Множину $M\subset \RR$ називають \textcolor{NavyBlue}{\textbf{\textit{множиною Лебегової міри нуль}}},\newline якщо $\forall \varepsilon > 0$ існує послідовність відрізків $I_n,\: n \in \NN$, не виключаючи порожню множину, що
    \begin{enumerate}
        \item $M\subset \bigcup\limits_{n=1}^\infty I_n$
        \item сумарна довжина $I \leqslant  \varepsilon \text{, тобто} \Sum_{n=1}^\infty \mu (I_n)\leqslant \varepsilon$, де $I_n = [a_n;\: b_n]$, $\mu(I_n)=b_n-a_n$
    \end{enumerate}
\end{definition}


\begin{theorem}[\textcolor{Maroon}{\textbf{\textit{Рімана (критерій інтегровності за Ріманом)}}}]
    Нехай $f(x)$ -- обмежена на $[a;\:b]$ функція. Тоді $f(x)$ інтегровна за Ріманом $\iff$ множина її точок  розриву має Лебегову міру нуль.
\end{theorem}
\textcolor{Maroon}{\textbf{\textit{{Наслідок.}}}}  Нехай $f(x)$ і $g(x)$ інтегровні за Ріманом на $[a;\:b]$ функції, тоді $f(x) \pm g(x)$, $f(x) \cdot g(x)$ інтегровні за Ріманом на $[a;\:b]$ функції.


\begin{theorem}[\textcolor{Maroon}{\textbf{\textit{лінійність інтеграла Рімана}}}]
    Нехай $f(x)$ і $g(x)$ -- функції, інтегровні за Ріманом на $[a;\:b]$, тоді
    \begin{enumerate}
        \item $ \Int_a^b (f+g) \,\mathrm{d} x = \Int_a^b f \,\mathrm{d} x + \Int_a^b g \,\mathrm{d} x $
        \item $ \Int_a^b c f \,\mathrm{d} x = c \Int_a^b f \,\mathrm{d} x $
    \end{enumerate}
\end{theorem}


\begin{theorem}[\textcolor{Maroon}{\textbf{\textit{адитивність інтеграла Рімана відносно області інтегрування}}}]
    Нехай $f(x) $ -- функція, інтегровна за Ріманом на $[a;\:b]$, тоді
    $ \forall c \in [a;\:b]: \ \exists \Int_a^c f(x) \,\mathrm{d} x,\: \Int_c^b f(x) \,\mathrm{d} x \text{, що}  \Int_a^b f(x) \,\mathrm{d} x = \Int_a^c f(x) \,\mathrm{d} x + \Int_c^b f(x) \,\mathrm{d} x $
\end{theorem}


\begin{theorem}[\textcolor{Maroon}{\textbf{\textit{монотонність інтеграла Рімана}}}]
    Нехай $f(x)$ і $g(x)$ -- функції, інтегровні за Ріманом на $[a;\:b]$ i $\forall x \in [a;\:b] :$ $f(x) \leqslant g(x)$. Тоді $\Int_a^{b}f(x) \,\mathrm{d} x \leqslant \Int_a^b g(x) \,\mathrm{d} x$
\end{theorem}
\textcolor{Maroon}{\textbf{\textit{{Наслідки.}}}}
\begin{enumerate}
    \item Якщо $f(x)\geqslant 0$, то $\Int_{a}^{b}f \,\mathrm{d} x \geqslant 0 $
    \item Якщо $ m \leqslant f(x) \leqslant M\ \forall x \in [a;\:b]$, то  $m(b-a) \leqslant \Int_a^b f         \,\mathrm{d} x \leqslant M(b-a)$    
    \item $\left | \Int_{a}^{b}f(x)\,\mathrm{d} x \right |\leqslant \Int_{a}^{b}\left | f(x) \right | \,\mathrm{d}     x$
\end{enumerate}


\begin{theorem}[\textcolor{Maroon}{\textbf{\textit{неперервність інтеграла Рімана зі змінною верхнею межею}}}]
    Нехай $f(x) $ -- інтегровна за Ріманом на $[a;\:b]$ функція, тоді $F_r(t)=(R)\Int_{a}^{t}f(x)\,\mathrm{d} x, \ t \in (a;\:b]$ є непепервною на $[a;\:b]$.
\end{theorem}


\begin{theorem}[\textcolor{Maroon}{\textbf{\textit{диференційовність інтеграла Рімана зі змінною  верхнею межею}}}]
    Нехай $f(x)$ -- функція, інтегровна за Ріманом на  $[a;\:b]$, тоді $ F_r(t)=\Int_{a}^{t}f(x)\,\mathrm{d} x,\: t \in (a;\:b]$ є диференційовною в усіх точках $t_0 \in (a;\:b]$, де функція $f(x)$ є неперервною.
\end{theorem}


\textcolor{Maroon}{\textbf{\textit{{Наслідки.}}}}
\begin{enumerate}
    \item Будь-яка неперервна на відрізку функція має первісну
    \item $f \:-$ неперервна на відрізку функція, тоді $\ (N\text{-}L)\Int_{a}^{b}f(x)\,\mathrm{d} x = (R)\Int_{a}^{b}f(x)\,\mathrm{d} x$
\end{enumerate}


\begin{theorem}
    Нехай $g(x) $ -- інтегровна за Ріманом на $[a;\:b]$ функція i $ m \leqslant g(x) \leqslant M $. Тоді $\exists \mu \in [m;\:M]$, що $\Int_a^b g(x)\,\mathrm{d} x = \mu(b-a),\ $  а якщо $g(x)\:- $  неперервна функція, то $\exists \xi \in [a;\:b]: \ \Int_a^b g(x)\,\mathrm{d} x=g(\xi )(b-a)$
\end{theorem}


\begin{theorem}[\textcolor{Maroon}{\textbf{\textit{про середнє}}}]
    Нехай $f(x)$ і $g(x)\: -$ функції, інтегровні за Ріманом на  $[a;\:b]$.\newline Якщо $f(x)\geqslant 0$ на $[a;\:b]$ і $ m\leqslant g(x)\leqslant M \text{, то } \exists \mu \in [m;\:M]$, що виконується
    $\Int_a^b f(x)g(x)\,\mathrm{d} x= \mu \Int_a^b f(x)\,\mathrm{d} x$
    \newline Крім цього, якщо $g(x)\: - $  неперевна функція, то $\exists \xi \in [a;\:b] \text{, що } \mu=g(\xi).$
\end{theorem}


\subsection{\large{Застосування Інтеграла Рімана}}


\begin{definition}
    Функцію виду $[a;\:b]\overset{F}{\rightarrow}\RR$ називають \textcolor{NavyBlue}{\textbf{\textit{функцією проміжку}}}.
\end{definition}


\begin{definition}
    Функцію проміжку F називають \textcolor{NavyBlue}{\textbf{\textit{адитивною}}}, якщо $\forall a< c< b$ виконується \[F([a;\:b])=F([a;\:c])+F([c;\:b])\]
\end{definition}


\begin{theorem}
    Нехай $F $ -- адитивна функція проміжку та існує $f$, інтегровна за Ріманом на $[a;\:b]$, так що \[ \forall [\alpha ;\:\beta ]\subset [a;\:b]: \ m(\beta -\alpha) \leqslant F([\alpha ;\:\beta ])\leqslant M(\beta -\alpha) \text{, де } m=\underset{{x\in [\alpha\: \beta]}}{\inf} F(x),\ M=\underset{{x\in [\alpha\: \beta]}}{\sup} F(x)\] \[\text{Тоді } F([a;\:b])=(R)\Int_a^b f(x)\,\mathrm{d} x \]
\end{theorem}    



\newpage
\section{\Large{Ряди}}
$a_1 + a_2 + \ldots + a_n + \ldots \: -$ ряд, $a_i \in \RR$. \\

Розглянемо послідовність $a_n \in \RR$:
\begin{equation} 
    % one can turn off the numeration by using \begin{equation*}
    % same works with any other numbered environment
    \label{ep:1.1}
    \begin{aligned}
        S_1 &= a_1\\
        S_2 &= a_1 + a_2\\
        S_3 &= a_1 + a_2 + a_3 \\
        \ldots \\
        S_n &= a_1 + a_2 + \ldots + a_n \\
        \ldots
    \end{aligned}
\end{equation}

\begin{definition}
    Послідовність $S_n$ називають  \textcolor{NavyBlue}{\textbf{\textit{послідовністю часткових сум}}}.
\end{definition}

\begin{definition}
    $a_n$ називають \textcolor{NavyBlue}{\textbf{\textit{загальним членом ряду}}.}
    \end{definition} 

\begin{definition}
    Нехай $a_1 + a_2 + a_3 + \ldots = \Sum_{n=1}^\infty a_n \: -$  числовий ряд. Кажуть, що цей  \textcolor{NavyBlue}{\textbf{\textit{ряд збігається до числа}}} $S \in \RR$, якщо послідовність $S_n = a_1 + a_2 + a_3 + \ldots + a_n = \Sum_{k=1}^n a_k$ збігається до числа $S$.
\end{definition}

\begin{theorem}[\textcolor{Maroon}{\textbf{\textit{{{необхідна умова збіжності}}}}}]
    Нехай $\Sum_{n=1}^\infty a_n \: -$ збіжний числовий ряд. Тоді $a_n \to 0$.
\end{theorem}

% \let\biconditional\leftrightarrow
% \newcommand{\gt}{\textgreater}
% \newcommand{\lt}{\textless} 

\begin{theorem}[\textcolor{Maroon}{\textbf{\textit{{{критерій Коші, критерій збіжності ряду}}}}}]
    % one can use \xrightarrow[below]{above} to produce long \rightarrow with superscript and subscript
    Числовий ряд $\Sum_{n = 1}^\infty a_n$  є збіжним $\iff $ \newline $\Sum_{k=n}^m a_k \xrightarrow[n, m \to \infty]{} 0$, тобто 
   $ \forall \varepsilon > 0 \: \, \exists N \ \forall n, m \geqslant N, \: m \geqslant n:  \, \: \left|\Sum_{k=n}^m a_k\right| < \varepsilon$.
\end{theorem}

\begin{theorem}[\textcolor{Maroon}{\textbf{\textit{{{про арифметичні дії зі збіжними рядами}}}}}]
    Нехай $\Sum_{n=1}^\infty a_n$,  $\Sum_{n=1}^\infty b_n \: -$ збіжні числові ряди. \newline Тоді також будуть збіжними ряди
    \begin{enumerate}
        \item $\Sum_{k=1}^\infty (a_k \pm b_k)$
        
        \item $\Sum_{k=1}^\infty \lambda a_k \: $, $\lambda \in \mathbb{R}$
    \end{enumerate}
\end{theorem}

\subsection{\large{Ряди з невід'ємними членами}}

\begin{theorem}[\textcolor{Maroon}{\textbf{\textit{{{критерій збіжності ряду з невід'ємними членами}}}}}]
    % use \iff to produce \Leftrightarrow in the "if and only if" meaning
    % \geqslant stand for Greater or Equal, \leqslant for Less or Equal, for > and < one can simply use the corresponding keys on the keyboard
    % oh, btw, u can use \geslant and \leqslant slant to produce Russian-style comparators
    Нехай $a_n \geqslant 0$, тоді ряд $\Sum_{n=1}^\infty a_n$ є збіжним  $\iff$ послідовність $S_n = a_1 + a_2 + a_3 + \ldots + a_n$ обмежена.
\end{theorem}

\begin{theorem}[\textcolor{Maroon}{\textbf{\textit{ознаки порівняння №1}}}]
    Нехай $a_n, \: b_n \: -$ числові послідовності та $0\leqslant  a_n \leqslant  b_n$. Тоді 
    
    \begin{enumerate}
        \item якщо ряд $\Sum_{n=1}^\infty b_n$ збіжний, то ряд  $\Sum_{n=1}^\infty a_n$ теж збіжний
        
        \item якщо ряд $\Sum_{n=1}^\infty a_n$ розбіжний, то ряд  $\Sum_{n=1}^\infty b_n$ теж розбіжний
    \end{enumerate}
\end{theorem}

\newpage
\begin{theorem}[\textcolor{Maroon}{\textbf{\textit{{{ознаки порівняння №2}}}}}]
    Нехай $a_n, \: b_n > 0$. Якщо $\dfrac{a_{n + 1}}{a_n} \leqslant  \dfrac{b_{n+1}}{b_n}$, то 
    \begin{enumerate}
        \item зі збіжності ряду $\Sum_{n=1}^\infty b_n$ випливає збіжність ряду $\Sum_{n=1}^\infty a_n$
        
        \item з розбіжності ряду $\Sum_{n=1}^\infty a_n$ випливає розбіжність ряду $\Sum_{n=1}^\infty b_n$
    \end{enumerate}
\end{theorem}

\begin{theorem}[\textcolor{Maroon}{\textbf{\textit{{{ознака Маклорена-Коші}}}}}]
    Нехай $f(x) \: -$ неперервна на $[1; \: + \infty)$ функція. Якщо
    \begin{enumerate} 
        \item $f(x) \geqslant 0$, $\forall x \in [1; \: + \infty)$% you do not usually place a white space between sign and a number
        
        \item $f(x) \: -$ монотонно спадна функція, $\forall x \in [1; \: + \infty)$
    \end{enumerate}
    
    тоді ряд $\Sum_{n=1}^\infty f(n)$ збігається або розбігається одночасно з $\Lim_{m \to \infty} \Int_1^m f(x) \, \mathrm{d}x$.
\end{theorem}
%can it be written next to the ukr text? not in the center PERFECT 
% $ $ -- inline math mode
% \[ \] or $$ $$ -- math mode on the separate line ok
% btw, u can press Command+S or Command+Return to recompile file (there is no need to press the button itself) 

\subsection{\large{Еталонні ряди}}
 \begin{enumerate}
            \item $\Sum_{n=1}^\infty q^{n} \: -$ геометрична прогресія. 
          Збіжний при $|q| < 1 $, розбіжний при  $|q|  \geqslant  1$.
           \item $\Sum_{n=1}^\infty \dfrac{1}{n^{\lambda}}  \: - $ степеневий ряд.
           Збіжний при $\lambda >1 $, розбіжний при  $\lambda  \leqslant  1$.
           \item $\Sum_{n=2}^\infty \dfrac{1}{n  \ln^{\lambda}{n}}$. Збіжний при $\lambda >1 $, розбіжний при  $\lambda  \leqslant  1$.
           \item $\Sum_{n=2}^\infty \dfrac{1}{n  \ln{n}  (\ln{\ln{{n}}})^{\lambda}}$. Збіжний при $\lambda >1 $, розбіжний при  $\lambda  \leqslant  1$.
          
       
       \end{enumerate}

\subsection{\large{Ознаки збіжності рядів з невід'ємними членами}}

\begin{theorem}[\textcolor{Maroon}{\textbf{\textit{{{ознака Коші}}}}}]
Нехай $a_n \geqslant  0$ i  $ \exists \overline{\Lim_{n \to \infty}}
 \sqrt[\leftroot{-1}\uproot{2}\scriptstyle n]{a_n} =r $. Якщо $r < 1 $ ряд $\Sum_{n=1}^\infty a_n \: - $ збіжний, якщо \newline $r > 1 \: - $ розбіжний.
\end{theorem}

\begin{theorem}[\textcolor{Maroon}{\textbf{\textit{{{ознака Д'Аламбера}}}}}]
Нехай $a_n > 0$ i   $\exists {\Lim_{n \to \infty}} \dfrac{a_{n+1}}{a_n}=r $. Якщо $r < 1 $ ряд $\Sum_{n=1}^\infty a_n - $ збіжний, якщо \newline $r > 1 \: - $ розбіжний.
\end{theorem}

\begin{theorem}[\textcolor{Maroon}{\textbf{\textit{{{ознака Раабе}}}}}]

Нехай $a_n > 0$ і $\exists \Lim_{n \to \infty} n\left(\dfrac{a_n}{a_{n+1}}-1\right)=r \in \mathbb{\overline{R}}$. Якщо $r > 1 $ ряд $\Sum_{n=1}^\infty a_n \: - $ збіжний, якщо $ r < 1 \: - $ розбіжний.
\end{theorem}

\begin{theorem}[\textcolor{Maroon}{\textbf{\textit{{{ознака Гаусса}}}}}]

Нехай $a_n > 0$ i $\forall n $ виконується рівність $\dfrac{a_n}{a_{n+1}}= 1 + \dfrac{\lambda}{n}+\dfrac{O(1)}{n^{1+\varepsilon}}$, де $\varepsilon = const,\varepsilon > 0$. Тоді
       \begin{enumerate}
           \item при $\lambda > 1 $ ряд  $\Sum_{n=1}^\infty a_n $ збіжний
           \item при $\lambda \leqslant  1 $ ряд  $\Sum_{n=1}^\infty a_n $ розбіжний
       \end{enumerate}
       
\end{theorem}

\subsection{\large{Знакозмінні ряди}}
\begin{definition}
  Ряд $\Sum_{n=1}^\infty a_n$ називають \textcolor{NavyBlue}{\textbf{\textit{абсолютню збіжним}}}, якщо збігається ряд  $\Sum_{n=1}^\infty |a_n|$.
\end{definition}

\begin{theorem}
Якщо $\Sum_{n=1}^\infty a_n$ збігається абсолютно, то  $\Sum_{n=1}^\infty a_n \: -$ збіжний ряд.
\end{theorem}
\begin{definition}
    Ряд $\Sum_{n=1}^\infty a_n$ називають \textcolor{NavyBlue}{\textbf{\textit{умовно збіжним}}}, якщо він збігається, але не абсолютно. 
\end{definition}




\begin{theorem}[\textcolor{Maroon}{\textbf{\textit{{{перетворення Абеля}}}}}]
Нехай $a_n, \: b_n \: - $ послідовності. Тоді


\centerline{$\Sum_{k=1}^n a_k(b_k-b_{k-1}) = a_nb_n-a_1b_0-
\Sum_{k=1}^{n-1}b_k(a_{k+1}-a_k)$}
\end{theorem}

\subsection{\large{Ознаки збіжності знакозмінних рядів}}

\begin{theorem}[\textcolor{Maroon}{\textbf{\textit{{{ознака Діріхле}}}}}]
Нехай $a_n, \: b_n \: -$ числові послідовності. Якщо  
       \begin{enumerate}
           \item $a_n - $ монотонна послідовність 
           \item $a_n \rightarrow 0 $
           \item послідовність $B_n=b_1+b_2+...+b_n \: - \ $обмежена
       \end{enumerate}
 то $\Sum_{n=1}^\infty a_n  b_n \: - \ $збіжний ряд.

\end{theorem}
\begin{theorem}[\textcolor{Maroon}{\textbf{\textit{{{ознака Лейбніца}}}}}]
Нехай $a_n \: - $ числова послідовність. Якщо  
       \begin{enumerate}
           \item $a_n  \: - $ монотонна послідовність 
           \item $a_n \rightarrow 0 $
       \end{enumerate}
 то $\Sum_{n=1}^\infty(-1)^n  a_n \: - \ $збіжний ряд.


\end{theorem}
\begin{theorem}[\textcolor{Maroon}{\textbf{\textit{{{ознака Абеля}}}}}]
Нехай $a_n, \: b_n \: -$ числові послідовності. Якщо  
       \begin{enumerate}
           \item $a_n  \: - $ монотонна послідовність 
           \item $a_n  \: - $ обмежена послідовність (збіжна)
           \item  $\Sum_{n=1}^\infty b_n  \: - $ збіжний ряд
       \end{enumerate}
 то $\Sum_{n=1}^\infty a_n  b_n  \: - \ $збіжний ряд.


\end{theorem}

\subsection{\large{Структура умовно збіжних рядів}}
Нехай $\Sum_{n=1}^\infty a_n \: -$ умовно збіжний ряд. Тоді 
\begin{equation*}
a^+_n =  \begin{cases}
            a_n, & \text{якщо $a_n \geqslant 0 $} \\
            0, & \text{якщо $a_n < 0$} 
       \end{cases} \quad
a^-_n =  \begin{cases}
               a_n, & \text{якщо $a_n \leqslant  0 $} \\
            0, & \text{якщо $a_n > 0$} \\
      
       \end{cases}
\end{equation*}


\begin{theorem}
Нехай $\Sum_{n=1}^\infty a_n \: - $ умовно збіжний ряд. Тоді  $\Sum_{n=1}^\infty a_n^{+} = +\infty$ в $\mathbb{\overline{R}}$ і 
  $\Sum_{n=1}^\infty a_n ^{-}= -\infty$ в $\mathbb{\overline{R}}$.
\end{theorem}


\begin{theorem}[\textcolor{Maroon}{\textbf{\textit{{{Рімана}}}}}] 

Нехай $\Sum_{n=1}^\infty a_n \: -$ умовно збіжний ряд. Тоді  $\forall l \in 
\mathbb{\overline{R}}$ існує така перестановка доданків  $b_n = a_{k(n)}$ 
($b_n \: -$ така сама послідовніть $a_n$, але числа записані в іншому порядку), що  $\Sum_{n=1}^\infty b_n = l$.
\end{theorem}


\section{\Large{Невласні інтеграли}}

\begin{definition}
    Нехай $f \: -$ інтегровна за Ріманом функція на $[a; \:b]$ для всіх $ b>a,\; a \: -$ фіксоване. Якщо існує границя$ \Lim_{b \to +\infty} \Int_a^b f(x) \, \mathrm{d}x=I \in \mathbb{R}$ , то кажуть, що  \textcolor{NavyBlue}{\textbf{\textit{невласний інтеграл першого роду}}} $\Int_a^{+\infty} f(x) \, \mathrm{d}x \ $  \textcolor{NavyBlue}{\textbf{\textit{є збіжним. }}}
\end{definition}
\begin{theorem}[\textcolor{Maroon}{\textbf{\textit{{{критерій Коші}}}}}] 
$\Int_a^{+\infty} f(x) \, \mathrm{d}x \ $ є збіжним $\iff \Lim_{c, b \to +\infty} \Int_c^b f(x) \, \mathrm{d}x = 0$.
\end{theorem}

\begin{theorem}[\textcolor{Maroon}{\textbf{\textit{{{про арифметичні дії з невласними інтегралами}}}}}]  

\end{theorem}

\begin{enumerate} 
\item  $\Int_a^{+\infty} \alpha f(x) \, \mathrm{d}x = \alpha \Int_a^{+\infty} f(x) \, \mathrm{d}x $
\item $\Int_a^{+\infty} (f(x) \pm g(x)) \, \mathrm{d}x = \Int_a^{+\infty} f(x) \, \mathrm{d}x \pm \Int_a^{+\infty} g(x) \, \mathrm{d}x $
\end{enumerate}

\begin{theorem}[\textcolor{Maroon}{\textbf{\textit{{{ознаки порівняння}}}}}] 
Якщо $0 \leqslant  f(x) \leqslant  g(x) $ і 
     \begin{enumerate}
           \item $\Int_a^{+\infty} g(x) \, \mathrm{d}x \: - $  збіжний, то $\Int_a^{+\infty} f(x) \, \mathrm{d}x \: - $ збіжний 
           \item  $\Int_a^{+\infty} f(x) \, \mathrm{d}x \: - $ розбіжний, то 
            $\Int_a^{+\infty} g(x) \, \mathrm{d}x \: - $ розбіжний
       \end{enumerate}

\end{theorem}
 \textcolor{NavyBlue}{\textbf{\textit{Еталонні невласні інтеграли }}}
(збіжні при $p>1$, розбіжні при $p \leqslant  1$):

\begin{enumerate}
\item $\Int_a^{+\infty} \dfrac{1}{x^p} \, \mathrm{d}x $


\item $\Int_a^{+\infty} \dfrac{1}{x  \ln^px} \, \mathrm{d}x $


\item $\Int_a^{+\infty} \dfrac{1}{x  \ln x  \ln \ln^px} \, \mathrm{d}x $
\end{enumerate}

\begin{theorem}[\textcolor{Maroon}{\textbf{\textit{{{ознака Діріхле}}}}}] 
Нехай $f$ і $g \: -$ неперервні на $[a; \: +\infty)$ функції і 
       \begin{enumerate}
           \item $f \: - $ монотонна 
           \item $f(x) \rightarrow 0 $ при $x \to +\infty$
           \item $\Int_a^b g(x) \, \mathrm{d}x = G(b) \: - $ обмежена функція
       \end{enumerate}
Тоді $\Int_a^{+\infty} f(x)  g(x) \, \mathrm{d}x \: - $ збіжний невласний інтеграл першого роду. 
\end{theorem}
\newpage
\begin{theorem}[\textcolor{Maroon}{\textbf{\textit{{{ознака Абеля}}}}}] 
Нехай $f$ і $g \: -$ неперервні на $[a; \: +\infty)$ функції і 
       \begin{enumerate}
           \item $f \: - $ монотонна 
           \item $f \: - $ обмежена функція $\left( \exists \Lim_{x \to \infty} f(x)\right)$
           \item $\Int_a^{+\infty} g(x) \, \mathrm{d}x - $ збіжний інтеграл
       \end{enumerate}
Тоді $\Int_a^{+\infty} f(x)  g(x) \, \mathrm{d}x \: - $ збіжний невласний інтеграл першого роду. 

\end{theorem}

 \textcolor{NavyBlue}{\textbf{\textit{Зауваження!}}} Для невласних інтегралів НЕ виконується аналогічна необхідна умова збіжності рядів. 

\begin{definition}
    Нехай $f \: -$ неперервна функція на $(a; \: b]$. Якщо існує границя $ \Lim_{A \to a^+} \Int_A^b f(x) \, \mathrm{d}x=I \in \mathbb{R}$ , то кажуть, що  \textcolor{NavyBlue}{\textbf{\textit{невласний інтеграл другого 
    роду є збіжним}}}, тобто $\Int _a^b f(x) \, \mathrm{d}x = I$.
\end{definition}
\textcolor{NavyBlue}{\textbf{\textit{Зауваження!}}} Нехай $f \: - \ $неперервна фукція на  $(a; \: b)\cup (b; \: +\infty) $. Тоді під $\Int _a^{+\infty} f(x) \, \mathrm{d}x$ розуміють суму чотирьох невласних інтегралів (за умови, що всі невласні інтеграли $I_1, I_2, I_3, I_4$ збіжні). 

$I_1 = \Int _a^{\frac{a+b}{2}} f(x) \, \mathrm{d}x$, \;
$I_2 = \Int _{\frac{a+b}{2}}^b f(x) \, \mathrm{d}x$, \;
$I_3 = \Int _b^{b+1} f(x) \, \mathrm{d}x$,  \;
$I_4 = \Int _{b+1}^{+\infty} f(x) \, \mathrm{d}x$.

\textcolor{NavyBlue}{\textbf{\textit{Зауваження!}}} Заміною $t=\dfrac{1}{x-a\:}$  невласний інтеграл другого роду $ \Int_a^b f(x) \, \mathrm{d}x \ $ зводиться до невласного інтеграла першого роду ($x = a \: -$ особлива точка).

\section{\Large{Функціональні послідовності та ряди}}

\subsection{\large{Функціональні послідовності}}
\begin{definition}
$f_1(x), \: f_2(x), \: f_3(x), \: f_4(x), \: ... \: - $  \textcolor{NavyBlue}{\textbf{\textit{функціональна послідовність}}}.
\end{definition}

\begin{definition}
Нехай $f_n(x) \: - $ функціональна послідовність, $x \in A.$
Кажуть, що $f_n $ \textcolor{NavyBlue}{\textbf{\textit{поточково збігається}}} до функції $f$ на множині $A$, якщо  $\forall x \in A \;\; f_n(x) \rightarrow f(x), \; A \subset D_{f_n} $. Позначають $f_n \rightarrow f$.
\end{definition}

\begin{definition}
Нехай $\Sum_{n=1}^{\infty} f_n(x) \: - $ функціональний ряд, $x \in A.$
Кажуть, що $\Sum_{n=1}^{\infty} f_n(x) $ \textcolor{NavyBlue}{\textbf{\textit{поточково збігається}}} до функції $f(x)$ на множині $A$, якщо  $\forall x \in A \;\;  \Sum_{n=1}^{\infty} f_n(x) = f(x), \; A \subset D_{f_n} $. 
\end{definition}

\begin{definition}
Нехай $f_n(x) \: - $ функціональна послідовність, $x \in A,\; A \subset D_{f_n} $. 
Кажуть, що $f_n(x) $ \textcolor{NavyBlue}{\textbf{\textit{рівномірно збігається}}} до функції $f$ на множині $A$, якщо  
$ \underset{x \in A}{\sup}|f_n(x) - f(x)| \rightarrow 0$. Позначають $f_n \rightrightarrows f$.
\end{definition}

\begin{theorem}[\textcolor{Maroon}{\textbf{\textit{{{про зв'язок між рівномірною і поточковою збіжностями}}}}}] 
Нехай $f_n \rightrightarrows f$ на множині $A$. Тоді $f_n \rightarrow f$ на множині $A$. 
\end{theorem} 

\begin{theorem}[\textcolor{Maroon}{\textbf{\textit{{{необхідна умова рівномірної збіжності}}}}}] 
Нехай $f_n \rightrightarrows f$ на множині $A$, $f \: -$ обмежена. $\newline$ Тоді $f_n \: - $ рівномірно обмежена, тобто 



\centerline{$\exists c>0 \;\; \forall n \in \mathbb{N} \;\; \forall x \in A: \;\; |f_n(x)| \leqslant  c$}



 \hspace{184}$\exists c>0 \;\; \forall n \in \mathbb{N}: \;\;  \underset{A}{\sup}|f_n(x)| \leqslant c$
\end{theorem} 

\begin{theorem} [\textcolor{Maroon}{\textbf{\textit{{{про арифметичні дії з рівномірно збіжними функціональними послідовностями}}}}}] $\newline$
Нехай $f_n \rightrightarrows f$,  $g_n \rightrightarrows g$ на множині $A$, $f \: -$ обмежена. Тоді 
 \begin{enumerate}
           \item  $f_n \pm g_n \rightrightarrows f \pm g$
           \item $kf_n \rightrightarrows kf \: (k = const)$
       \end{enumerate}
\end{theorem} 

\begin{theorem}[\textcolor{Maroon}{\textbf{\textit{критерій Коші}}}]

\end{theorem} 
 Функціональна послідовність $f_n \: $ збігається рівномірно до функції $f$ на множині $A  \iff \underset{x \in A}{\sup} \left|f_n(x) - f_m(x)\right|  \rightarrow 0$.


\subsection{\large{Функціональні ряди}}

\begin{definition}
$\Sum_{n=1}^\infty f_n(x) \: -$  \textcolor{NavyBlue}{\textbf{\textit{функціональний ряд}}}.
\end{definition}

\begin{definition}
Кажуть, що \textcolor{NavyBlue}{\textbf{\textit{фунціональний ряд}}} $\Sum_{n=1}^\infty f_n(x)$  \textcolor{NavyBlue}{\textbf{\textit{збігається рівномірно}}} до функції $f(x)$ на $\newline$ множині $A$, якщо $S_n(x) = \Sum_{k=1}^n f_k(x) \underset{A}{\rightrightarrows }f(x)$.
\end{definition}

\begin{theorem}[\textcolor{Maroon}{\textbf{\textit{{{критерій Коші}}}}}] 
Функціональний ряд $\Sum_{n=1}^\infty f_n(x)$ збігається рівномірно на множині $A\iff$

\centerline{$\underset{x \in A}{\sup} \left|  \Sum_{k=n}^m f_k(x)\right| \xrightarrow[m, n \to +\infty]{} 0$}
\end{theorem} 

 \textcolor{NavyBlue}{\textbf{\textit{Зауваження!}}} ${\sup}\left|f(x)\right|$ часто позначають $\|f(x)\|$.
 
$\hspace{20cm}$
\textcolor{Maroon}{\textbf{\textit{{Наслідок(необхідна умова рівномірної збіжності функціонального ряду)}}}}
Якщо функціональний ряд $\Sum_{n=1}^\infty f_n(x) \: $ збігається рівномірно на множині $A$, то послідовність $\|f_n(x)\| \rightarrow 0$, тобто $\underset{x \in A}{\sup} \left|f_n(x)\right| \rightarrow 0$.

\begin{theorem}[\textcolor{Maroon}{\textbf{\textit{{{мажорантна ознака Вейєрштраcса}}}}}] 
Нехай $\Sum_{n=1}^\infty f_n(x) \: -$ функціональний ряд. Існує числова послідовність $\alpha_n$, що $\forall x \in A, \; n \in \mathbb{N}: \; |f_n(x)| \leqslant  \alpha_n$. Якщо ряд, утворений з цих чисел, збіжний $\left(\Sum_{n=1}^\infty \alpha_n \right)$, то 
 $\Sum_{n=1}^\infty f_n(x)$  збігається рівномірно на множині $A$.
 \end{theorem} 

\begin{theorem}[\textcolor{Maroon}{\textbf{\textit{{{ознака Діріхле}}}}}] 
Нехай $f_n(x)$ і $g_n(x) \: -$ функціональні послідовності 
       \begin{enumerate}
           \item $\forall x \in A \; \;f_n(x) \: - $ монотонна послідовність (за змінною $n$)
           \item $f_n \rightrightarrows 0 $ на $A$
           \item $G_n(x) = g_1(x) + g_2(x) +....+ g_n(x) \: - \ $рівномірно обмежена, тобто $\exists c > 0 \;\; \forall x \in A \;\; \forall n \in \mathbb{N}: \;\; \left| G_n(x) \right| \leqslant  c$
       \end{enumerate}
Тоді $\Sum_{n=1}^{\infty} f_n(x)  g_n(x) $ рівномірно збіжний на множині $A$.

\end{theorem} 
\begin{theorem}[\textcolor{Maroon}{\textbf{\textit{{{ознака Абеля}}}}}] 
Нехай $f_n(x)$ і $g_n(x) \: -$ функціональні послідовності 
       \begin{enumerate}
           \item $\forall x \in A \; \;f_n(x) \: - $ монотонна послідовність (за змінною $n$)
           \item $f_n - $ рівномірно обмежена послідовність, тобто $\exists c > 0 \;\; \forall x \in A \;\; \forall n \in \mathbb{N}: \;\; \left| f_n(x) \right| < c$
           \item  $\Sum_{n=1}^{\infty} g_n \rightrightarrows $ на множині $A$
       \end{enumerate}
Тоді $\Sum_{n=1}^{\infty} f_n(x)  g_n(x) $ рівномірно збіжний на множині $A$.

\end{theorem} 
\begin{theorem}[\textcolor{Maroon}{\textbf{\textit{{{про неперервність границі функціональної послідовності}}}}}] 
\end{theorem} 
Нехай $f_n \in C([a; \: b])$ і $f_n \rightrightarrows f$ на $[a; \:b]$. Тоді $f \in C([a; \: b])$.

\begin{theorem}[\textcolor{Maroon}{\textbf{\textit{{{про інтегрування границі функціональної послідовності}}}}}] 

\end{theorem} 
Нехай $f_n \in C([a; \: b])$ і $f_n \rightrightarrows f$ на $[a; \:b]$. Тоді $\Int_a^b f_n  \, \mathrm{d}x \rightarrow \Int_a^b f \, \mathrm{d}x $.
\begin{theorem}[\textcolor{Maroon}{\textbf{\textit{{{про диференціювання границі функціональної послідовності}}}}}] 
Нехай $f_n \in C^1([a; \: b])$. Якщо
       \begin{enumerate}
           \item $\exists x_0 \in [a; \: b] \; \;f_n(x_0) - $ збіжна послідовність
           \item $f_n' \rightrightarrows g $ на $[a; \: b]$
       \end{enumerate}
тоді $\forall x \in [a; \: b]\;\; f_n \rightarrow f$, функція $f \in C^1([a; \: b])$ і $f' = g$, тобто $f'_n \rightarrow f'$.
\end{theorem} 

\begin{theorem}[\textcolor{Maroon}{\textbf{\textit{{{про диференціювання границі функціонального ряду}}}}}] 
Нехай $f_n \in C^1([a; \: b])$. Якщо
       \begin{enumerate}
           \item $\exists x_0 \in [a; \: b] \; \; \Sum_{n=1}^{\infty} f_n(x_0) \: - $ збіжний ряд
           \item  $\Sum_{n=1}^{\infty} f'_n \rightrightarrows g $
       \end{enumerate}
тоді $\forall x \in [a; \: b]: \;\; \Sum_{n=1}^{\infty} f_n(x)= f(x)$, функція $f \in C^1([a; \: b])$ і $f' = g$, тобто   $\Sum_{n=1}^{\infty} f'_n (x)= \left(\Sum_{n=1}^{\infty} f_n(x)\right)' $.


\end{theorem} 


\subsection{\large{Степеневі ряди}}
$a_0 + a_1x+...+ a_nx^n+.... \: - $  \textcolor{NavyBlue}{\textbf{\textit{степеневий ряд. }}}

\begin{definition}
    Множина $I \subset \mathbb{R}$ , на якій збігається степеневий ряд $\Sum_{n=0}^{\infty} a_nx^n$,  називається \textcolor{NavyBlue}{\textbf{\textit{множиною \newline збіжності степеневого ряду}}}.
\end{definition}

\begin{theorem}[\textcolor{Maroon}{\textbf{\textit{формула Коші-Адамара}}}] 
$R \: - \ $радіус збіжності степеневого ряду. $R = \dfrac{1}{l},\;\; l = \overline{\Lim_{n \to \infty}} \sqrt[\leftroot{-1}\uproot{2}\scriptstyle n]{|a_n|}$.
\end{theorem}

\begin{theorem}[\textcolor{Maroon}{\textbf{\textit{властивості степеневого ряду}}}] $\Sum_{n=0}^\infty a_nx^n = f(x)$. $R \: - \: $радіус збіжності.
       \begin{enumerate}
       \item $\forall x \in (-R; \: R) \;\;f \: - $ неперервна в точці $x$
       \item  $\Sum_{n=0}^\infty a_nx^n$ збігається рівномірно на $[a; \: b]$, де $[a; \: b] \subset (-R; \: R)$
       \item $[a; \: b] \subset (-R; \:R), \Int_a^b f(x) \, \mathrm{d}x =  \Int_a^b \Sum_{n=0}^{\infty} a_nx^n \, \mathrm{d}x = \Sum_{n=0}^{\infty} a_n \left ( \dfrac{b^{n+1}-a^{n+1}}{n+1} \right )$ 
           \item $\forall x \in (-R; \: R) \;\; \exists f'(x)$. Нехай $x_0 \in (-R; \: R)$. Тоді $[a; \: b] \subset (-R; \:R),\; x_0\in (a; \: b) $
            \begin{enumerate}
            \item  $\Sum_{n=0}^\infty a_nx^n \: -$ збіжний на $[a; \: b]$
          \item  $\Sum_{n=0}^\infty (a_nx^n)' \: -$ рівномірно збіжний на $[a; \: b]$
            \end{enumerate}
           \item  $f \in C^{\infty}(-R; \: R)$
       \end{enumerate}
\end{theorem}



\begin{definition}
    Кажуть, що  \textcolor{NavyBlue}{\textbf{\textit{функцію}}} $f $ \textcolor{NavyBlue}{\textbf{\textit{ можна розкласти в степеневий ряд}}} на $[a; \: b]$, якщо існує \newline степеневий ряд $\Sum_{n=0}^\infty a_nx^n$, що збігається до функції $f$.
\end{definition}

\begin{theorem}[]
Якщо $f $ можна розкласти в степеневий ряд, то такий розклад єдиний. 

\end{theorem}


\begin{theorem}
Нехай $f \in C^{\infty}(-R; \: R)$. Якщо $\exists c, \: a \in \mathbb{R}  \; $, що $\left| f^{(k)}(x)\right| \leqslant  c  a^k \ \ \forall x \in (-R; \: R)$, то \newline \centerline{$f(x) = f(0) + \dfrac{f'(0)}{1!}  x + \dfrac{f''(0)}{2!}  x^2 + \dfrac{f'''(0)}{3!}  x^3 + ...$.}

\end{theorem}

\section{\Large{Інтеграли, залежні від параметра}}
\subsection{\large{Власні інтеграли, залежні від параметра}}


\begin{theorem}[\textcolor{Maroon}{\textbf{\textit{{{про неперевність власного інтеграла, залежного від параметра}}}}}] 
Нехай  $f \:  -$ неперервна функція на множині $ [a; \: b] \times [c; \:d]$. Тоді $I(\alpha) = \Int_a^b f(x,\alpha) \, \mathrm{d}x $ є неперервною функцією на $[c; \:d]$.
\end{theorem} 
\textcolor{Maroon}{\textbf{\textit{Наслідок:}}} Нехай  $f \in C([a; \: b] \times [c; \:d])$. Тоді $\forall \alpha_0 \in (c; \:d) \Lim_{\alpha \to \alpha_0} \Int_a^b f(x, \: \alpha) \, \mathrm{d}x =\Int_a^b \Lim_{\alpha \to \alpha_0}  f(x, \: \alpha) \, \mathrm{d}x = \Int_a^b f(x, \: \alpha_0) \, \mathrm{d}x$.

\begin{theorem}[\textcolor{Maroon}{\textbf{\textit{{{про диференціювання власного інтеграла, залежного від параметра}}}}}] 


\end{theorem} 

Нехай $f(x, \alpha), \dfrac{\partial f}{\partial \alpha} (x, \alpha) \in C([a; \: b] \times [c; \: d])$. Тоді функція $F(\alpha) = \Int_a^b f(x, \: \alpha) \, \mathrm{d}x $ є диференційованою на $[c; \:d]  $ і\\ $F'(\alpha) = \Int_a^b  \dfrac{\partial f}{\partial \alpha} (x, \: \alpha) \, \mathrm{d}x $, тобто $\dfrac{\partial}{\partial \alpha}\left( \Int_a^b f(x, \: \alpha) \, \mathrm{d}x \right)=  \Int_a^b  \dfrac{\partial f}{\partial \alpha} (x, \: \alpha) \, \mathrm{d}x$.

\begin{theorem}[\textcolor{Maroon}{\textbf{\textit{про інтегрування власного інтеграла, залежного від параметра}}}] 

\end{theorem} 
Нехай $f(x, \alpha) \in C([a; \: b] \times [c; \:d])$.
Тоді $\Int_c^d F(\alpha) \, \mathrm{d}\alpha= \Int_c^d \left(\Int_a^b f(x, \: \alpha) \, \mathrm{d}x \right) \mathrm{d}\alpha = \Int_a^b \left(\Int_c^d f(x, \: \alpha) \, \mathrm{d}\alpha \right) \mathrm{d}x $.

 \textcolor{NavyBlue}{\textbf{\textit{Зауваження! }}}Нехай $F(\alpha) = \Int_{\varphi(\alpha)}^{\psi(\alpha)} f(x, \: \alpha) \, \mathrm{d}x  \:  - $ інший тип власного інтеграла, залежного від параметра. Має місце твердження:


якщо $f, \: \dfrac{\partial f}{\partial \alpha}, \: \varphi ', \: \psi '\: - $ неперервні функції, то $F'(\alpha) = \Int_{\varphi(\alpha)}^{\psi(\alpha)} f'(x, \: \alpha) \, \mathrm{d}x + f(\psi(\alpha), \: \alpha) \psi'(\alpha) - f(\varphi(\alpha),\: \alpha)  \varphi'(\alpha) $.
\subsection{\large{Невласні інтеграли першого роду, залежні від параметра}}

$F(\alpha) = \Int_a^{+\infty} f(x, \: \alpha) \, \mathrm{d}x  \:  - $  \textcolor{NavyBlue}{\textbf{\textit{невласний інтеграл першого роду, залежний від параметра}}}

\begin{definition}
Кажуть, що невласний інтеграл, залежний від параметра, $ \Int_a^{+\infty} f(x, \: \alpha) \, \mathrm{d}x$ \textcolor{NavyBlue}{ \textbf{\textit{збігається поточково}}} на множині $M \ (\alpha \in M)$, якщо $\forall \alpha_0 \in M$ невласний інтеграл $ \Int_a^{+\infty} f(x, \: \alpha_0) \, \mathrm{d}x $ збіжний. 
\end{definition}

Якщо $ \Int_a^{+\infty} f(x, \: \alpha) \, \mathrm{d}x$  збігається поточково на множині $M$ , то на множині $M$ визначена функція $F(\alpha) = \Int_a^{+\infty} f(x, \: \alpha) \, \mathrm{d}x $.

\begin{definition}
Кажуть, що невласний інтеграл, залежний від параметра, $\Int_a^{+\infty} f(x, \: \alpha) \, \mathrm{d}x = F(\alpha) $  \textcolor{NavyBlue}{\textbf{\textit{збігається рівномірно}}} на множині $M \ ( \alpha \in M)$, якщо $\underset{\alpha \in M}{\sup} \left|  \Int_a^A f(x, \: \alpha) \, \mathrm{d}x - F(\alpha)\right| \xrightarrow[A \to +\infty]{} 0$.



\end{definition}

\begin{theorem}[\textcolor{Maroon}{\textbf{\textit{{{критерій Коші}}}}}] 
Нехай $f(x, \alpha) \in C([a; \: +\infty) \times M)$. Тоді невласний інтеграл, залежний від параметра, $\Int_a^{+\infty} f(x, \: \alpha) \, \mathrm{d}x$  збігається рівномірно на множині $M \iff \underset{\alpha \in M}{\sup} \left|  \Int_A^B f(x, \: \alpha) \, \mathrm{d}x\right| \xrightarrow[A, B \to +\infty]{} 0$.

\end{theorem} 


\begin{theorem}[\textcolor{Maroon}{\textbf{\textit{{{мажорантна ознака Вейєрштрасса}}}}}] 
Нехай $f(x, \alpha) \in C([a; \: +\infty) \times M)$. Якщо $\exists g(x)\in C([a; \: +\infty))$ така, що $\left| f(x, \: \alpha)\right| \leqslant  g(x) \ \forall(x, \: \alpha) \in [a; \: +\infty) \times M$ і $\Int_a^{+\infty} g(x) \, \mathrm{d}x$  збігається, то невласний інтеграл, залежний від праметра, $\Int_a^{+\infty} f(x, \: \alpha) \, \mathrm{d}x$  збігається рівномірно на множині $M$.

\end{theorem} 

\begin{theorem}[\textcolor{Maroon}{\textbf{\textit{{{ознака Діріхле}}}}}] 
Нехай $f(x, \: \alpha), \; g(x, \: \alpha) \:  \: - \ $неперервні функції на $[a; \: +\infty) \times [c; \:d]$. Якщо  


1) $\forall\alpha_0 \in [c; \:d]$ функція $f(x, \alpha_0)$  монотонна (за змінною $x$)

2) $f(x, \: \alpha) \rightrightarrows 0$, тобто $\forall \varepsilon>0 \:\; \exists A>a \:\; \forall x> A \:\; \forall \alpha \in [c; \:d]:\;\; \left| f(x, \alpha)-0 \right| < \varepsilon$

3) $G(A, \: \alpha) = \Int_a^A g(x, \: \alpha) \, \mathrm{d}x$ рівномірно обмежена, тобто $\exists C \; \forall A > a \; \forall\alpha \in [c; \:d]: \: \left| G(A, \: \alpha) \right| < C$ \newline
то невласний інтеграл, залежний від параметра, $\Int_a^{+\infty} f(x, \: \alpha)  g(x, \: \alpha) \, \mathrm{d}x$ збігається рівномірно на $[c;\: d]$.



\end{theorem} 
\begin{theorem}[\textcolor{Maroon}{\textbf{\textit{{{ознака Абеля}}}}}] 
Нехай $f(x, \: \alpha),\; g(x, \: \alpha) \: - \ $неперервні функції на $[a; \: +\infty) \times [c; \:d]$. Якщо  


1) $\forall\alpha_0 \in [c; \:d]$ функція $f(x, \alpha_0)$  монотонна (за змінною $x$)

2) $f(x, \alpha)$ рівномірно обмежена, тобто $\exists C>0 \:\; \forall x> a \:\; \forall \alpha \in [c; \:d]: \;\; \left| f(x, \: \alpha) \right| < C$


3) інтеграл $\Int_a^{+\infty} g(x, \: \alpha) \, \mathrm{d}x$ рівномірно збігається на $[c;\:d]$ \newline
то невласний інтеграл, залежний від параметра, $\Int_a^{+\infty} f(x, \: \alpha)  g(x, \: \alpha) \, \mathrm{d}x$ збігається рівномірно на $[c;\: d]$.
\end{theorem} 

\begin{theorem}[\textcolor{Maroon}{\textbf{\textit{про неперервність невласного інтеграла, залежного від параметра}}}] 
\end{theorem} 
 Нехай $f(x, \alpha) \: - \ $неперервна функція на $[a; \: +\infty) \times [c; \:d]$. Якщо невласний інтеграл $\Int_a^{+\infty} f(x, \: \alpha) \, \mathrm{d}x$ збігається рівномірно на $[c; \: d]$, то 
 $F(\alpha) = \Int_a^{+\infty} f(x, \: \alpha) \, \mathrm{d}x \: - $ неперервна функція на $[c; \: d]$.
 \newpage
\begin{theorem}[\textcolor{Maroon}{\textbf{\textit{про диференціювання невласного інтеграла, залежного від параметра}}}] 
$\newline$
Нехай $f(x, \: \alpha)$ визначена на множині $[a; \: +\infty) \times [c; \:d]$. Якщо

1) $f, \: f'_\alpha  \: -$ неперервні функції на $[a; \: +\infty) \times [c; \:d]$


2)  $ \exists \alpha_0 \in [c; \: d]$, що $\Int_a^{+\infty} f(x, \: \alpha_0) \, \mathrm{d}x$ збігається


3) $\Int_a^{+\infty} f'_\alpha(x, \: \alpha) \, \mathrm{d}x \rightrightarrows \; G(\alpha)$ на $[c; \: d]$, то 

\hspace{45}1.  $\forall \alpha \in [c; \: d]: \Int_a^{+\infty} f(x, \: \alpha) \, \mathrm{d}x$ збіжний,
$F(\alpha) = \Int_a^{+\infty} f(x, \: \alpha) \, \mathrm{d}x$ 

\hspace{45}2. $F \in  C^1([c; \: d])$  і $F'(\alpha) = G(\alpha)
=\Int_a^{+\infty} f'_\alpha(x, \: \alpha) \, \mathrm{d}x$, тобто
$\left( \Int_a^{+\infty} f(x, \: \alpha) \, \mathrm{d}x   \right)'_\alpha =  \Int_a^{+\infty} f'_\alpha(x, \: \alpha) \, \mathrm{d}x $
\end{theorem} 

\begin{theorem}[\textcolor{Maroon}{\textbf{\textit{про інтегрування невласного інтеграла, залежного від параметра}}}] 
$\newline$ Нехай $f(x, \: \alpha) \: - \ $ неперервна функція на $[a; \: +\infty) \times [c; \:d]$. Якщо $\Int_a^{+\infty} f(x, \: \alpha) \, \mathrm{d}x$ збігається рівномірно на $[c; \:d] $, то 

\centerline{ $\Int_c^d \left(\Int_a^{+\infty} f(x, \: \alpha) \, \mathrm{d}x \right) \mathrm{d}\alpha = \Int_a^{+\infty} \left(\Int_c^d f(x, \: \alpha) \, \mathrm{d}\alpha \right) \mathrm{d}x $}

\end{theorem} 

\subsection{\large{Невласні інтеграли другого роду, залежні від параметра}}
 $F(\alpha) =  \Int_a^b f(x, \: \alpha) \, \mathrm{d}x  \: -$  \textcolor{NavyBlue}{\textbf{\textit{невласний інтеграл другого роду, залежний від параметра}}}.
 
 
$(x, \: \alpha) \in (a; \: b] \times [c; \: d], \; x=a \: - \:$ особлива точка.

Заміною $t=\dfrac{1}{x-a\:}$ невласний інтеграл другого роду, залежний від параметра, $ \Int_a^b f(x, \alpha) \, \mathrm{d}x \ $ зводиться до невласного інтеграла першого роду, залежного від параметра.

\subsection{\large{Важливі невласні інтеграли}}

       \begin{enumerate}
           \item \textcolor{NavyBlue}{\textbf{\textit{Інтеграл Діріхле}}}   $ \ I=\Int_0^{+\infty} \dfrac{\sin({\alpha x})}{x} \, \mathrm{d}x=\dfrac{\pi}{2}, \ \alpha > 0$
           \item \textcolor{NavyBlue}{\textbf{\textit{Інтеграл Фруллані}}}   $ \ I=\Int_0^{+\infty} \dfrac{f(bx)-f(ax)}{x} \, \mathrm{d}x=f(0)  \ln{\dfrac{a}{b}}, \ \alpha > 0$
           \item \textcolor{NavyBlue}{\textbf{\textit{Інтеграл Пуассона}}} $ \ I=\Int_0^{+\infty} e^{-x^2} \, \mathrm{d}x=\dfrac{\sqrt \pi}{2}$
           \item \textcolor{NavyBlue}{\textbf{\textit{Інтеграл Френеля}}}  $ \ I=\Int_0^{+\infty} \sin({x^2}) \, \mathrm{d}x
=\Int_0^{+\infty} \cos({x^2}) \, \mathrm{d}x = \dfrac{\sqrt \pi}{2\sqrt{2}}$
       \end{enumerate}
\textcolor{NavyBlue}{\textbf{\textit{Зауваження!}}} Функції
$y=\dfrac{\sin(x)}{x} \; (x > 0), \; y=e^{-x^2}, \;y=\sin({x^2}), \;y=\cos({x^2})$
неперервні і мають первісні, але вони не виражаються через елементарні функції. 
\newpage
\subsection{\large{Ейлерові інтеграли}}
       \begin{enumerate}
           \item \textcolor{NavyBlue}{\textbf{\textit{Гамма-функція}}}
           $\Gamma (s) = \Int_0^{+\infty}  e^{-x} x^{s-1} \, \mathrm{d}x$
           
           Властивості  $\Gamma (s) $:
          \begin{enumerate}
           \item    $D(\Gamma) = (0; \: +\infty)$
          \item $\Gamma (1) = 1 $
          \item $\Gamma (s+1) = s  \Gamma (s)$
          \item $\Gamma (n+1) = n!, \ n \in \mathbb{N}$
          \item $\forall[a; \: b], \: a, \: b > 0 \Int_0^{+\infty} e^{-x} x^{s-1} \, \mathrm{d}x \ $    збігається рівномірно на $[a; \: b]$
          \item  $\Gamma (s)$ неперервна функція на  $(0; \: +\infty)$
          \item  $\Gamma (s)  \in C^{\infty}((0; \: +\infty))$
          \item  $\Gamma '  (s) = \Int_0^{+\infty} e^{-x} x^{s-1} \ln(x) \, \mathrm{d}x$
    
       \end{enumerate}
       
       
       Формула доповнення: $\Gamma(\alpha)  \Gamma (1-\alpha) = \dfrac{\pi}{\sin({\pi \alpha})}$
       
       
       
       
       Формула Стірлінга: \(\Gamma(s+1) = \sqrt{2\pi s}  s^{s}   e^{\displaystyle- s+ \displaystyle \frac{\Theta(s)}{12s}}\), \: де $\Theta(s) \in (0; \; 1)$
           \item \textcolor{NavyBlue}{\textbf{\textit{$B$-функція}}} $B(\alpha, \: \beta) = \Int_0^1 x^{\alpha-1} (1-x)^{\beta-1} \, \mathrm{d}x $
           
            Властивості  $B(\alpha, \: \beta) $:
          \begin{enumerate}
          \item    $D(B) = (0; \: +\infty) \times (0; \:+\infty)$ 
         \item $B(\alpha, \: \beta) = B(\beta, \: \alpha)$
          \item $B(\alpha, \: \beta) = \dfrac{\Gamma(\alpha)
\Gamma(\beta)}{\Gamma(\alpha+\beta)}$
        \end{enumerate}
       \end{enumerate}



\newpage
\section{\Large{Інтеграл на брусі}}

\begin{definition}
    Число $I$ називають\textcolor{NavyBlue}{\textbf{\textit{ інтегралом функції}}} $f$\textcolor{NavyBlue}{\textbf{\textit{ на брусі}}} $D$, якщо \newline $\forall \varepsilon > 0 \;  \exists \delta >0 \ \forall P \ \forall \xi: |d_p| < \delta \Rightarrow  |S(P, \: \xi) - I| < \varepsilon. \newline$
    позначають: $I = \Int_D f(x, \: y) \, \mathrm{d}D = \Int_D f(x, \: y) \, \mathrm{d}x \,  \mathrm{d}y = \iint\limits_{D} f(x, \: y) \, \mathrm{d}x \, \mathrm{d}y$ 
\end{definition}


\begin{theorem}[\textcolor{Maroon}{\textbf{\textit{властивості інтеграла Рімана на брусі}}}]
\end{theorem}
    \begin{enumerate}
        \item $f(x, \: y) = $ const $\Rightarrow \Int_D C \, \mathrm{d}x \, \mathrm{d}y = C\mu(D) \;\;\;\;\; (\mu \;- $ міра $D$ в $\mathbb{R}^n)$
        
        \item якщо $f$ і $g \: - $ інтегровні функції на $D$, то $\forall c_1, c_2 \in \mathbb{R}:
        \newline {\Int_D(c_1f + c_2g) \, \mathrm{d}D = c_1\Int_D f \, \mathrm{d}D + c_2 \Int_D g \, \mathrm{d}D}$
        
        \item якщо $f$ і $g \: - $ інтегровні функції на $D$ і $f(x, \: y) \leqslant g(x, \: y) \ \forall (x, \: y) \in D$, то \newline $\Int_D f(x, \: y) \, \mathrm{d}x \, \mathrm{d}y \leqslant \Int_D g(x, \: y) \, \mathrm{d}x \, \mathrm{d}y$ 
        
        \item якщо $D = D_1 \cup D_2$, де $D_1, D_2, D \: -$ бруси, $D_1$ та $D_2$ не мають спільних комірок \newline і $f \: - $ інтегровна на $D$, то $\Int_D f \, \mathrm{d}D = \Int_{D_1} f \, \mathrm{d}D_1 + \Int_{D_2} f \, \mathrm{d}D_2$
    \end{enumerate}

\begin{theorem}[\textcolor{Maroon}{\textbf{\textit{{{Лебега, критерій інтегровності на брусі}}}}}]
    Нехай $D \: - $ брус, $f \: -$ обмежена функція на $D$, \newline множина точок розривів якої має лебегову міру нуль. Тоді $f$ інтегровна на брусі $D$. 
\end{theorem}


\begin{theorem}[\textcolor{Maroon}{\textbf{\textit{{{Фубіні №1}}}}}]
    Нехай $f \: - $ неперервна на брусі $D = [a_1, \: b_1]\times[a_2; \: b_2]$. Тоді $\newline$
  
    $\centerline{\Int_D f(x, \: y) \, \mathrm{d}x \, \mathrm{d}y = \Int_{a_1}^{b_1}\left(\Int_{a_2}^{b_2} f(x, \: y) \, \mathrm{d}y\right) \mathrm{d}x= \Int_{a_2}^{b_2}\left(\Int_{a_1}^{b_1} f(x, \: y) \, \mathrm{d}x\right) \mathrm{d}y}$
\end{theorem}


\begin{definition}
    Множину $M \subset \mathbb{R}^2$ називають \textcolor{NavyBlue}{\textbf{\textit{циліндричною}}}, якщо існують функції $\varphi$ та $\psi$, визначені на $[a; \: b]$, що $\varphi(x) \leqslant \psi(x) \ \forall x \in [a; \: b]$ та $M = \{(x; \: y)| \: x \in [a; \: b], \: \varphi(x) \leqslant y \leqslant \psi(x)\}$.
\end{definition}


\begin{theorem}[\textcolor{Maroon}{\textbf{\textit{{{Фубіні №2}}}}}]
    Нехай $D = \{(x; \: y)| \: x \in [a; \: b], \: \varphi(x) \leqslant y \leqslant \psi(x)\} \: - $ циліндрична множина, \newline а функція $f$ визначена та неперервна на $D$. Тоді \newline
    $\centerline{\Int_D f(x, \: y) \, \mathrm{d}x \, \mathrm{d}y = \Int_{a}^{b}\left(\Int_{\varphi(x)}^{\psi(x)} f(x, \: y) \, \mathrm{d}y\right) \mathrm{d}x}$
\end{theorem}


\section{\Large{Заміна змінних в кратному інтегралі}}
\begin{theorem}[\textcolor{Maroon}{\textbf{\textit{{{про за заміну змінної в кратному інтегралі}}}}}]
    Нехай $\Omega \subset \mathbb{R}^2 \: -$  компакт, функція $g: \mathbb{R}^2 \to \mathbb{R}^2, \newline g(u, \: v) = (x(u, \: v), \: y(u, \: v)) $ встановлює взаємно однозначну відповідність (бієкцію) між $\Omega $ і $D$. Якщо 
    \begin{enumerate}
        \item $g \in C^1(\Omega)$
        \item $f \in C(D)$
        \item  $\det \begin{pmatrix} x'_u & y'_u \\ x'_v & y'_v \end{pmatrix} \neq 0$ в $\Omega$, то 
    \end{enumerate}
     $\centerline{$\Int_D f(x, \: y) \, \mathrm{d}x \, \mathrm{d}y =\Int_\Omega f(x(u, \: v), \: y(u, \: v)) \left | \det \begin{pmatrix} x'_u & y'_u \\ x'_v & y'_v \end{pmatrix}(u, \: v) \right | \, \mathrm{d}u \, \mathrm{d}v $}$
\end{theorem}


\subsection{\large{Полярні координати}}
        \begin{cases}  
        x = r \cos{\varphi} \\
        y = r \sin{\theta} \\
    \end{cases}, де $\;\;\;$
    \begin{cases}  
        0 \leqslant r \\
        0 \leqslant \varphi \leqslant 2\pi \\ 
    \end{cases}
\newline
Визначник переходу від декартових координат $(x, \: y)$ до полярних координат $(\varphi, \: r)$ дорівнює \\ 
$\mu = \left | \det \dfrac{D(x, \: y)}{D(\varphi, \: r)} \right |= r$, \;\;\;
$\dfrac{D(x, \: y)}{D(\varphi, r)} = $
\begin{pmatrix} 
    x'_\varphi & y'_\varphi \\
    x'_r & y'_r \\ 
\end{pmatrix}


\subsection{\large{Узагальнені полярні координати}}
\begin{cases}  
        x = a r \cos^\alpha{\varphi} \\
        y = b r \sin^\alpha{\varphi} \\
    \end{cases}, де $\;\;\;$
    \begin{cases}  
        0 \leqslant r \\
        0 \leqslant \varphi \leqslant \dfrac{\pi}{2} \\ 
    \end{cases}
    $a > 0, \: b > 0$
\newline
$\mu = \left | \det \dfrac{D(x, \: y)}{D(\varphi, \: r)} \right |= a b\alpha r \cos^{\alpha-1}{\varphi}\sin^{\alpha-1}{\varphi}$, \;\;\;
$\dfrac{D(x, \: y)}{D(\varphi, r)} = $
\begin{pmatrix} 
    x'_\varphi & y'_\varphi \\
    x'_r & y'_r \\ 
\end{pmatrix}


\subsection{\large{Сферичні координати}}
    \begin{cases}  x = \rho \cos{\theta} \cos{\varphi} \\ y = \rho \cos{\theta} \sin{\varphi} \\ z = \rho \sin{\theta}  \end{cases}, де \;\;\;\;  \begin{cases} \;\;\;\;  0 \leqslant \rho \\  -\dfrac{\pi}{2} \leqslant \theta \leqslant \dfrac{\pi}{2} \\  \; \;\;\: 0 \leqslant \varphi \leqslant 2 \pi  \end{cases}


\newline
Визначник переходу від декартових координат $(x, \: y, \: z)$ до сферичних координат $(\rho, \: \theta, \: \varphi)$ дорівнює \\
\hspace{0.1}

 $\mu = \left | \det \dfrac{D(x, \: y, \: z)}{D(\rho,\theta,\varphi)} \right |=\rho^2 \cos{\theta}, \;\;\;
$\dfrac{D(x, \: y, \: z)}{D(\rho, \: \theta, \: \varphi)} = $
\begin{pmatrix} x'_\rho & x'_\theta & x'_\varphi \\ y'_\rho & y'_\theta & y'_\varphi \\ z'_\rho & z'_\theta & z'_\varphi \end{pmatrix}


\subsection{\large{Узагальнені сферичні координати}}
    \begin{cases}  x = a \rho \cos^\alpha {\theta} \cos^\beta {\varphi} \\ y = b \rho \cos^\alpha{\theta} \sin^\beta {\varphi} \\ z = c \rho \sin^\alpha{\theta}  \end{cases}, де \;\;\; \begin{cases} x \geqslant 0 \\ y \geqslant 0  \\  z \geqslant  0\end{cases}, \;\;\; \begin{cases} 0 \leqslant \theta \leqslant \dfrac{\pi}{2} \\   0 \leqslant \rho \\ 0 \leqslant \varphi \leqslant \dfrac{\pi}{2}  \end{cases} \\
 
\hspace{0.05}

$\mu = \left | \det \dfrac{D(x, \: y, \: z)}{D(\rho, \: \theta, \: \varphi)} \right |=abc \alpha \beta \rho^2 \cos^{2\alpha -1}{ \theta } \sin^{\alpha -1}{ \theta } \cos^{\beta -1}{\varphi} \sin^{\beta -1}{ \varphi }$


\subsection{\large{Циліндричні координати}}
    \begin{cases}  x = \rho \cos{\varphi} \\ y = \rho \sin{\varphi} \\ z = h \end{cases}, де \;  \begin{cases}\;  0 \leqslant \rho \\  \; 0 < \varphi \leqslant 2 \pi \\  \; h \in \mathbb{R}  \end{cases}\\

\hspace{0.1}

$\mu = \left | \det \dfrac{D(x, \: y, \: z)}{D(\rho, \: \theta, \: \varphi)} \right |= \rho$
 

\subsection{\large{Сферичні координати в  $\mathbb{{R}}^n$}}
 \begin{cases}  x_1 \; \; \; \, = \; \rho \cos{\theta_1} \cos{\theta_2} \dots \cos{\theta_{n-2} \cos{\varphi} \\ x_2 \; \; \; \, = \; \rho \cos{\theta_1} \cos{\theta_2} \dots \cos{\theta_{n-2} \sin{\varphi} \\ \dots \\ x_{n-1} = \; \rho \cos{\theta_1} \sin{\theta_2} \\ x_n \; \; \; \, = \; \rho \sin{\theta_1}  \end{cases}, де \;  \begin{cases} \;\;\;\;\;\;\;\;\;\;\;\;\:\; 0 \leqslant \rho \\  \; - \dfrac{\pi}{2} \leqslant \theta_1, \theta_2, \dots , \theta_{n-2} \leqslant \dfrac{\pi}{2} \\  \; \;\;\;\;\;\;\;\;\;\;\;\; 0 \leqslant \varphi \leqslant 2 \pi  \end{cases} \\
 
 \hspace{0.1}
 
 $\mu = \left | \det \dfrac{D(x_1, \: \dots, \: x_n)}{D(\rho, \: \theta_1, \: \dots,\: \theta_{n-2}, \: \varphi)} \right |=\rho^{n-1} \cos^{n-2}{ \theta_1 } \cos^{n-3}{ \theta_2}\dots \cos{ \theta_{n-2}}$
 


\section{\Large{Криволінійні інтеграли}}
\subsection{\large{Криволінійний інтеграл першого роду}}
$l:\begin{cases}  x=x(t) \\ y=y(t)  \end{cases} - $ неперервно-диференційовані функції, $t \in [t_0; \: t_1]$ \\
$\vv{v}(t) = (x'(t), \: y'(t)), \;\; \left | \vv{v}(t) \right | \neq 0, \;\; \left | \vv{v}(t) \right | =  \sqrt{(x'(t))^2 + (y'(t))^2}  $\\

$l=\Int_{t_0}^{t_1} \sqrt{(x'(t))^2 + (y'(t))^2} \, \mathrm{d}t \: -$ формула знаходження довжини кривої в декартових координатах \\

 $I=\Int_{t_0}^{t_1} f(x(t), \: y(t)) \left | v(t) \right | \mathrm{d}t \: - $
 \textcolor{NavyBlue}{\textbf{\textit{криволінійний інтеграл першого роду}}}.\\
Має таке позначення: $I= \Int_{l} f(x, \: y) \, \mathrm{d}l$\\
Криволінійний інтеграл по замкненому контуру позначається так: $$I= \oint\limits_{l} f(x, \: y) \, \mathrm{d}l$$

\subsection{\large{Криволінійний інтеграл другого роду}}

$A \: - $ робота по переміщенню з точки A в точку B. $\vv{F}=(f_1(x, \: y), \: f_2(x, \: y)), \;f_i: \mathbb{R}^2 \rightarrow \mathbb{R}, \; f_i \in C(\mathbb{R}^2) $\\
 $\Delta A= |  \vv{F}(t) | \cos{ \varphi} \left |  \vv{v}(t) \right |  \Delta t = ( \vv{F}(t), \: \vv{v}(t)) \Delta t $ \\
 $A = \Int_{t_0}^{t_1} (\vv{F}(t), \vv{v}(t)) \, \mathrm{d}t = \Int_{t_0}^{t_1} (f_1(x(t), \: y(t))x'(t) + f_2(x(t), \: y(t))y'(t)) \, \mathrm{d}t \:- \textcolor{NavyBlue}{\textbf{\textit{криволінійний інтеграл другого роду}}}. \newline$
 Має таке позначення: $I= \Int\limits_{l_{op.}} f_1 \mathrm{d}x + f_2 \mathrm{d}y$
 
\subsection{\large{Формула Гріна}}
Нехай $D\: - $ циліндрична множина (або розбивається на циліндричні множини), $f_1, f_2 \in C^1 (  \mathbb{R}^2).$ Тоді \\
$$\oint\limits_{C_{op.}} f_1 \, \mathrm{d}x + f_2 \, \mathrm{d}y = \iint\limits_{D} \left ( \dfrac{\partial{f_2}}{\partial{x}}-\dfrac{\partial{f_1}}{\partial{y}}\right) \mathrm{d}x \, \mathrm{d}y$$


\begin{theorem}
Нехай $D \: -$ циліндрична множина, $D \subset \mathbb{R}^2,\: f=(f_1, f_2), \: f_i:\mathbb{R}^2 \rightarrow \mathbb{R}, \: f_i \in C^1(D). \newline $ Тоді такі умови є рівносильними:
    \begin{enumerate}
            \item  Для будь-якого замкненого контура $C$ в $D:$ $$\oint\limits_{C_{op.}} f_1 \, \mathrm{d}x + f_2 \, \mathrm{d}y = 0 $$
         \item Нехай $A, B \in D$. Тоді для будь-якої кривої $l$, що з'єднує т. $A$ і т. $B$ \,($A-$початок): $\Int_{l}} f_1 \, \mathrm{d}x + f_2 \, \mathrm{d}y = const$
        \item $\exists u: \mathbb{R}^2 \rightarrow \mathbb{R}$ (повний диференціал, потенціал): $\mathrm{d}u = f_1 \, \mathrm{d}x + f_2 \, \mathrm{d}y$ в $D$ 
        \item $\dfrac{\partial{f_1}}{\partial{y}}=\dfrac{\partial{f_2}}{\partial{x}}$ в $D$
    \end{enumerate}
\end{theorem}


\section{\Large{Поверхневі інтеграли}}

\subsection{\large{Поверхневі інтеграли першого роду}}
Нехай система $\;$
\begin{cases}
        x = x(u, \:v)\\
        y = y(u, \:v)\\
        z = z(u, \:v)\\
\end{cases}
задає поверхню $S$ (параметрично), а функція $f(x, \: y, \:z)$ густину в кожній точці.

Маса $M$ поверхні $S$ знаходиться таким чином:\\
\hspace{0.3}  
$\vv{\varphi_u} = (x'_u, \: y'_u, \:z'_u)$ та $\vv{\varphi_v} = (x'_v, \: y'_v, \:z'_v)$
\\[0.15cm]
$E = {\left| \vv{\varphi_u}\right|}^2 = (x'_u)^2 + (y'_u)^2 + (z'_u)^2;$ 
$\;\;\;\; G = {\left|\vv{\varphi_v}\right|}^2 = (x'_v)^2 + (y'_v)^2 + (z'_v)^2;$
$\;\;\;\;F = (\vv{\varphi_u}, \: \vv{\varphi_v}) = x'_u x'_v + y'_u y'_v + z'_u x'_v$
\\[0.15cm]
Тоді $M = \Int_D f(x(u,v), \:y(u,v), \: z(x(u,v), \:y(u,v)) \sqrt{E G - F^2} \,\mathrm{d}u \,\mathrm{d}v \;$ називають \textcolor{NavyBlue}{\textbf{\textit{поверхневим інтегралом першого роду}}} і позначають $\Int_S f \, \mathrm{d}S$

Якщо поверхню задано явно, тобто $z = z(x,\: y), \;(x,\:y) \in D$, то
$M = \Int_D f(x, \:y, \: z(x, \:y)) \sqrt{(z'_x)^2 + (z'_y)^2 + 1} \,\mathrm{d}x \,\mathrm{d}y$


\subsection{\large{Поверхневі інтеграли другого роду}}

Нехай $F = (f_1,\: f_2,\: f_3)\:-$ вектор швидкості, $f_i: \mathbb{R}^3\to \mathbb{R}, \; f_i \in C(\mathbb{R}^3). \; S_{op.} \: -$ орієнтована поверхня.

Число $m = \Int_{S_{op.}} f_1\,\mathrm{d}y\,\mathrm{d}z + f_2\,\mathrm{d}z\,\mathrm{d}x + f_3\,\mathrm{d}x\,\mathrm{d}y \;$ називають \textcolor{NavyBlue}{\textbf{\textit{поверхневим інтегралом другого роду}}}

Якщо $S_{op.}$ задано параметрично, то
$m =
    \Int_D f_1\begin{vmatrix}
     y'_u& z'_u\\ 
     y'_v& z'_v
    \end{vmatrix}
+
     f_2\begin{vmatrix}
     z'_u & x'_u\\ 
     z'_v & x'_v 
    \end{vmatrix}
+
    f_3\begin{vmatrix}
     x'_u& y'_u\\ 
     x'_v& y'_v
    \end{vmatrix}
\, \mathrm{d}u \, \mathrm{d}v
$

Якщо $S_{op.}$ задано явно, то $m = \Int_D f_1(-z'_x) + f_2(-z'_y) + f_3 \, \mathrm{d}x \, \mathrm{d}y$


\begin{theorem}[\textcolor{Maroon}{\textbf{\textit{{{формула Остроградського-Гаусса}}}}}]
    Нехай $V\:-$ циліндричне тіло, обмежене замкненою \newline поверхнею $S. \; F=(f_1, \: f_2, \: f_3)\: -$ векторне поле, $f_i \in C^1(\mathbb{R}^3), \; \vv{n}\:-$ вектор зовнішньої нормалі до поверхні. Тоді \\
    
    \centerline{\Int_{S_{op.}} (\vv{F}, \: \vv{n})  \,\mathrm{d}S = \Int_V \dfrac{\partial f_1}{\partial x} + \dfrac{\partial f_2}{\partial y} + \dfrac{\partial f_3}{\partial z} \,\mathrm{d}x \,\mathrm{d}y \,\mathrm{d}z}
   
\end{theorem}


\begin{theorem}[\textcolor{Maroon}{\textbf{\textit{{{формула Стокса}}}}}]
    Нехай $S\:-$ деяка поверхня, межею якої є замкнений контур $C. \newline F=(f_1, \: f_2, \: f_3)\: -$ векторне поле, $f_i \in C^1(\mathbb{R}^3), \; \vv{n}\:-$ вектор нормалі. Тоді має місце рівність: \\
    \centerline{\Int_{S_{op.}} (\operatorname{rot} \vv{F}, \vv{n})  \,\mathrm{d}S = \oint\limits_{C_{op.}} f_1 \,\mathrm{d}x +f_2 \,\mathrm{d}y +f_3 \,\mathrm{d}z}


    Ліву частину формули можна переписати так:\\
 \centerline{\Int_{S_{op.}} \left(\dfrac{\partial f_3}{\partial y}-\dfrac{\partial f_2}{\partial z}\right) \mathrm{d}y \,\mathrm{d}z +\left(\dfrac{\partial f_1}{\partial z}-\dfrac{\partial f_3}{\partial x}\right) \mathrm{d}z \,\mathrm{d}x + \left(\dfrac{\partial f_2}{\partial x}-\dfrac{\partial f_1}{\partial y}\right) \mathrm{d}x \,\mathrm{d}y }
\end{theorem}


\newpage
\section{\Large{Теорія міри}}

\subsection{\large{Класи множин}}
\begin{definition}
    Сукупність множин $\mathcal{H} \subset 2^X $ називається \textcolor{NavyBlue}{\textbf{\textit{напівкільцем}}}, якщо
    \begin{enumerate}
        \item $\forall A, B \in \mathcal{H} \Rightarrow  A \cap B \in \mathcal{H}$
        \item $\forall A, B \in \mathcal{H} \Rightarrow \exists C_1, C_2, \dots ,C_n \in \mathcal{H}: \; A \setminus B = \displaystyle\bigsqcup\limits_{i = 1}^n C_i$
    \end{enumerate}
\end{definition}


\begin{definition}
    Нехай $\mathcal{H}\:-$ півкільце ($\mathcal{H} \subset 2^X$). Функція $\mu:\mathcal{H} \rightarrow [0; \: + \infty] $ називається  
    \textcolor{NavyBlue}{\textbf{\textit{мірою}}} на $\mathcal{H}$, якщо
    \begin{enumerate}
        \item $\forall A \in \mathcal{H}, \; \mu (A) \geqslant 0 \;\; (+\infty \geqslant 0 \: -$ невід'ємність)  
        \item $\forall A_1, A_2, .\,.\,. \in \mathcal{H}, \; \displaystyle\bigsqcup\limits_{i = 1}^\infty  A_i \in \mathcal{H}: \mu\left(\displaystyle\bigsqcup\limits_{i = 1}^\infty  A_i\right) =  \Sum_{i=1}^{\infty} \mu (A_i) \;\; (\sigma$-адитивність)
    \end{enumerate}
\end{definition}


\subsection{\large{Продовження міри на кільце}}
\begin{definition}
    Нехай $X \neq \varnothing$. Сукупність множин $\mathcal{K} \subset 2^X $ називається \textcolor{NavyBlue}{\textbf{\textit{кільцем}}}, якщо
    \begin{enumerate}
        \item $\forall A, B \in \mathcal{K} \Rightarrow  A \cap B \in \mathcal{K}$
        \item $\forall A, B \in \mathcal{K} \Rightarrow  A \setminus B \in \mathcal{K}$
    \end{enumerate}
\end{definition}

\begin{theorem}
    Нехай $\mathcal{K} \:-$кільце ($\mathcal{K} \subset 2^X$). Тоді $\mathcal{K} \:-$ півкільце.
\end{theorem}

\begin{definition}
    $M \subset 2^X$. Кільце, яке позначають $\mathcal{K} (M)$, називається  \textcolor{NavyBlue}{\textbf{\textit{породженим кільцем}}}, якщо
    \begin{enumerate}
        \item $M \subset \mathcal{K}(M)$
        \item $\forall \mathcal{K}\:-$ кільце: $M \subset \mathcal{K}$ виконується, що $\mathcal{K}(M) \subset \mathcal{K}$, тобто $\mathcal{K}(M)\:-$найменше кільце, що містить $M$
    \end{enumerate}
\end{definition}


\begin{theorem}[\textcolor{Maroon}{\textbf{\textit{{{про породжене кільце}}}}}]
    Нехай $\mathcal{H}\:-$півкільце, $\mathcal{H} \subset 2^X$. Тоді породжене кільце утворюється так: \centerline{$\mathcal{K}( \mathcal{H})= \left\{\displaystyle\bigsqcup\limits_{i = 1}^m C_i | \; C_i \in \mathcal{H}, \; m \in \mathbb{N}\right\}$}
\end{theorem}


\subsection{\large{Продовження міри на алгебрі}}
\begin{definition}
    Нехай $\mathcal{A}\:-$ кільце (півкільце), $\mathcal{A} \subset 2^X$. Якщо $X \in \mathcal{A},$  то $\mathcal{A}$ називається \newline \textcolor{NavyBlue}{\textbf{\textit{алгеброю (півалгеброю)}}}.
\end{definition}


\subsection{\large{Підхід Лебега}}


\begin{definition}
    Сукупність множин $\sigma \mathcal{K} \subset 2^X$ називають $\sigma$-кільцем, якщо
    \begin{enumerate}
        \item $\sigma \mathcal{K}\:-\:$кільце
        \item $\forall A_i \in \sigma \mathcal{K}, \;\; i \in \mathbb{N}
      	\Rightarrow  \displaystyle\bigcup\limits_{i = 1}^{\infty} A_i \in \sigma \mathcal{K}$
    \end{enumerate}
\end{definition}


\begin{definition}
    Сукупність множин $\sigma \mathcal{A} \subset 2^X$ називають $\sigma$-алгеброю, якщо
    \begin{enumerate}
        \item $\sigma \mathcal{A} - \sigma$-кільце
        \item $X \in \sigma \mathcal{A}$
    \end{enumerate}
\end{definition}


\begin{definition}
   Нехай $\mathcal{H} \subset 2^X-$ сукупність множин. $\sigma$-кільце (позн. $\sigma \mathcal{K}(\mathcal{H}))$ називають породженим сукупністю $\mathcal{H}$, якщо
    \begin{enumerate}
        \item $\mathcal{H} \subset \sigma \mathcal{K}(\mathcal{H}) $
        \item якщо $\sigma \mathcal{K}_1-\sigma$-кільце, що містить $\mathcal{H}$, то $\sigma \mathcal{K}(\mathcal{H}) \subset \sigma \mathcal{K}_1$ 
    \end{enumerate}
\end{definition}
Аналогічно визначається породжена $\sigma$-алгебра.
\begin{definition}
   Функцію  $\lambda^*: 2^X \rightarrow \mathbb{\overline{R}}, \; D(\lambda^*) = 2^X $ називають \textcolor{NavyBlue}{\textbf{\textit{зовнішнью мірою}}} (абстрактною зовнішнью мірою), якщо
    \begin{enumerate}
        \item $\lambda^*(\varnothing) = 0 $
        \item $\lambda^*(A) \geqslant 0 \;\; \forall A \subset X$
        \item $\forall A, A_i \subset X \;\; A \subset \displaystyle\bigcup\limits_{i = 1}^{\infty} A_i: \;\;\; \lambda^*(A) \leqslant \Sum_{i=1}^{\infty} \lambda^*(A_i)$
    \end{enumerate}
\end{definition}

\begin{theorem}[\textcolor{Maroon}{\textbf{\textit{{{властивості зовнішньої міри}}}}}]
    Нехай $\lambda^*-$ зовнішня міра. Тоді
     \begin{enumerate}
        \item $\forall A, B \;\; A \subset B:\;\; \lambda^*(A) \leqslant \lambda^*(B)$
        \item $\forall A, B \;\; A \subset B:\;\; \lambda^*(A \cup B) \leqslant \lambda^*(A) + \lambda^*(B)$
    \end{enumerate}
\end{theorem}

\begin{definition}[Каратеодорі]
     Нехай $\lambda^*-$ зовнішня міра над $X$. Множину $A \subset X$ називають  \textcolor{NavyBlue}{\textbf{\textit{вимірною}}}, якщо $\forall B \subset X$ виконується рівність:\\
     \centerline{$\lambda^*(B) = \lambda^*(B \cap A) + \lambda^*(B \cap \overline{A})$}
\end{definition}

\begin{theorem}
    Нехай $\lambda^*-$ зовнішня міра над $X$. Тоді сукупність $S$ всіх вимірних множин утворює $\sigma$-алгебру та звуження $\lambda^*$ на $S$ є мірою.
\end{theorem}

\begin{theorem}
    Нехай $\mu \;-$ міра на кільці $\mathcal{K},\; \mu^*\;-$ зовнішня міра, породжена $\mu$, $S \; -$ клас вимірних множин. Тоді
     \begin{enumerate}
        \item $\mu^*|_{\mathcal{K}} = \mu$
        \item $\mathcal{K} \subset S$
    \end{enumerate}
\end{theorem}

\[
\begin{tikzpicture}
    \path 
      (0,0) rectangle (10,6.5) [draw]
      (0.4,0.4) rectangle (9.6,5.5) [draw]
      (0.8,0.8) rectangle (9.2,4.5) [draw]
      (1.2,1.2) rectangle (8.8,3.5) [draw]
      (1.6,1.6) rectangle (8.4,2.5) [draw]
      
      (3.65,6) node[right]{Булеан $2^X,\; \mu^*$ }
      (1.85,5) node[right]{Клас вимірних множин $S,\; \mu = \mu^*|_S $} 
      (1.45,4) node[right]{Породжена $\sigma-$алгебра $\mathcal{A}=\mathcal{A}(\mathcal{K}), \; \mu = \mu^*|_S $} 
      (2.37,3) node[right]{Породжене кільце $\mathcal{K}=\mathcal{K}(\mathcal{H}), \; \mu$}
      (3.73,2) node[right]{Півкільце $\mathcal{H},\; \mu$} 
\end{tikzpicture}
\]


\subsection{\large{Міра Лебега}}
\begin{definition}
Зафіксуємо півкільце $\mathcal{H}=\left\{[a;\:b)\;| \; a,b \in \mathbb{R},\; a<b\right\} \cup \left\{ \varnothing \right\} $ \\
Породжене кільце $\mathcal{K}=\mathcal{K}(\mathcal{H})$\\
$\mu^* \;-$ зовнішня міра, породжена мірою $\mu$ на $\mathcal{K}$\\
$2^\RR \xrightarrow{\mu^*} [0; \; +\infty)$\\
$S \;-$ $\sigma$-алгебра вимірних множин $(\sigma a(K) \subset S)$\\
$\mu (\varnothing) = 0, \;\; \mu ([a; \;b)) = b-a$\\
$\mu = \mu^*|_S \;-$ міра Лебега.\\
$\mu \left(\displaystyle\bigsqcup\limits_{i = 1}^n [a_i; \; b_i)\right)=\Sum_{i=1}^{n}\mu \left([a_i; \; b_i)\right)$\\
Тоді $S\;-$ клас вимірних множин за Лебегом. Клас $\mathcal{A}=\mathcal{A}(\mathcal{K})$ називають  \textcolor{NavyBlue}{\textbf{\textit{класом борельових множин}}}.\\
Позначають $\mathscr{B}$        
\end{definition}
\newpage
\centerline{\textbf{Приклади борельових множин}}
  \begin{enumerate}
        \item $[a; \;b) \in \mathscr{B}$
        \item $(a; \;b) = \displaystyle\bigcup\limits_{i = 1}^{\infty} [a + \dfrac{1}{i}; \; b) \in \mathscr{B}$
        \item $\left\{ a \right\} = [a; \;b) \setminus (a; \; b) \in \mathscr{B}$
        \item $(a; \;b] = (a; \;b) \cup  \left\{ b \right\} \in \mathscr{B}$
        \item  $A\;-$ відкрита множина,  $A = \displaystyle\bigcup\limits_{i = 1}^{\infty} (a_i; \; b_i) \in \mathscr{B}$
        \item  $A\;-$ замкнена множина,  $A = \mathbb{R} \setminus B \in \mathscr{B} \;\; (B\;- $ відкрита $)$
        \item $K_3\;-$ множина Кантора (замкнена), $K_3 \in \mathscr{B}$
        \item $A \; - $ зліченна множина,  $A =\displaystyle\bigcup\limits_{i = 1}^{\infty} \left\{a_i \right\} \in \mathscr{B}$ 
    \end{enumerate}
\centerline{\textbf{Схема поширення міри}} 
\begin{enumerate}
    \item Визначаємо міру $\mu$ на півкільці $\mathcal{H}$
    \item Продовжуємо $\mu$ на породжене кільце
$\mathcal{K}=\mathcal{K}(\mathcal{H})$
\item Будуємо $\mu^*\;-$ зовнішня міра, породженна $\mu$
    \item Звужуємо $\mu^*$ на клас вимірних множин $S$, де $\mu^*$ стає мірою на $S$
\end{enumerate}
    
\subsection{\large{Інтеграл Лебега}}


\begin{definition}
        Нехай $y = f(x) \: -$ обмежена функція та існують міри множин $\{x \, | \; y_i \, \leqslant f(x) < y_{i+1} \}$. \newline
        Тоді $s = \Sum_{i=0}^{n-1} y_i \, \mu\left(\{x \, | \; y_i \leqslant f(x) < y_{i+1} \}\right)$ називають \textcolor{NavyBlue}{\textbf{\textit{нижньою сумою Лебега}}}.
\end{definition}


\begin{definition}
        Нехай $y = f(x) \: -$ обмежена функція та існують міри множин $\{x \, | \; y_i \, \leqslant f(x) < y_{i+1} \}$. \newline
        Тоді $S = \Sum_{i=0}^{n-1} y_{i+1} \, \mu\left(\{x \, | \; y_i \leqslant f(x) < y_{i+1} \}\right)$ називають \textcolor{NavyBlue}{\textbf{\textit{верхньою сумою Лебега}}}.
\end{definition} 

Ключова властивість сум Лебега:
0 \leqslant S - s = \Sum_{i=0}^{n-1} (y_{i+1}-y_i) \, \mu\left(\{x \, | \; y_i \leqslant f(x) < y_{i+1} \}\right) \leqslant d\,\Sum_{i=0}^{n-1} \mu\left(\{x \, | \; y_i \leqslant f(x) < y_{i+1} \}\right)=\]  
\[=d  \mu\left(\displaystyle\bigsqcup\limits_{i = 0}^{n-1}  \{x \, | \; y_i \leqslant f(x) < y_{i+1} \}\right) = d\mu([a;\:b]) \xrightarrow[d\to0]{}0, \;\; d = \max_i(y_{i+1}-y_i}) \]


\subsection{\large{Вимірні функції}}


\begin{definition}
        \textcolor{NavyBlue}{\textbf{\textit{Функція}}} $f: \RR \to \RR$ називається \textcolor{NavyBlue}{\textbf{\textit{вимірною (вимірною за Лебегом)}}}, якщо \newline $\forall a \in \RR$ множина $\{f < a\}$ вимірна за Лебегом. 
\end{definition}


\begin{definition}
        Кажуть, що якась \textcolor{NavyBlue}{\textbf{\textit{властивість}}} (рівність або нерівність) \textcolor{NavyBlue}{\textbf{\textit{виконується майже скрізь}}}, якщо множина тих значень, для яких вона не виконується, є множиною Лебегової міри нуль.
\end{definition}
\begin{theorem}
  Нехай $f: \RR \rightarrow \RR \;-$ вимірна функція, $k\;-$ стала. Тоді
  \begin{enumerate}
      \item $f + k$
      \item $fk$
      \item $|f|}$
      \item $f^2$
      \item $\dfrac{1}{f}$
  \end{enumerate}
  є вимірними.
 \newpage
\end{theorem}
\textcolor{NavyBlue}{\textbf{\textit{Зауваження!}}} Якщо $f: \RR \rightarrow \RR \;-$ вимірна функція, то $\forall a \in \RR$ множини
  \begin{enumerate}
      \item $\left\{ f<a \right\}$
      \item $\left\{ f>a \right\}$
      \item $\left\{ f \leqslant a \right\} = \overline{\left\{ f>a \right\}}= \displaystyle\bigcap\limits_{m = 1}^{\infty} \left\{ f<a+\dfrac{1}{m} \right\} $
      \item $\left\{ f \geqslant a \right\}$
  \end{enumerate}
є вимірними. І навпаки, якщо, наприклад, $\forall a \left\{ f \leqslant a \right\} \;-$ вимірна множина, то $f \;-$ вимірна функція. \\
Справді, $\left\{ f<a \right\} =\displaystyle\bigcup\limits_{m = 1}^{\infty} \left\{ f \leqslant a-\dfrac{1}{m} \right\} $

\begin{theorem}
   Нехай $f: \RR \rightarrow \RR,\; g: \RR \rightarrow \RR \;-$ вимірні функції. Тоді
   \begin{enumerate}
       \item $f \pm g$
       \item $f \cdot g$
       \item $\dfrac{f}{g}$
   \end{enumerate}
  є вимірними функціями.
\end{theorem}    
\newpage

Якщо Ви знайшли помилку, чи є якісь інші запитання $-$ звертайтесь до Телеграм:
\begin{enumerate}
    \item @angelcoder
    \item @YouNeverFindMe
\end{enumerate}
Даний файл може періодично оновлюватись. \newline
Більш актуальну версію можете дивитися тут: \newline
https://drive.google.com/drive/folders/1US4yXTwLFWrZD0sVRuHWJr591KyDUeB$\_$


 
\end{document}